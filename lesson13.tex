\documentclass[11pt,a4paper]{article}

% PAQUETES
\usepackage[T1]{fontenc}%
\usepackage[utf8]{inputenc}%

\usepackage[english]{babel}
\usepackage{amsmath}
\usepackage{amsthm}
\usepackage{amsfonts}
\usepackage[%left=1.54cm,right=1.54cm,top=1.54cm,bottom=1.54cm
    margin=1in, includefoot,
]{geometry}
\usepackage{xfrac}  
\usepackage{tikz-cd}
\usepackage{enumerate}
\usepackage{amsfonts}
\usepackage{amssymb}
\usepackage{tcolorbox}
\usepackage{rotating}
\usepackage{mathpazo}
% \usepackage{charter}
\usetikzlibrary{babel}
\usepackage{listings}
\usepackage{amssymb}
\usepackage{extarrows}
\usepackage{makeidx}
\usepackage{graphicx}
\usepackage{multirow}
\usepackage{tikz-cd}
\usepackage{tasks}
\usepackage{xcolor}
\usepackage{mathrsfs} % 2024-11-04

% Christian
\usepackage{enumitem,etoolbox,titlesec}


%OPERADORES

\DeclareMathOperator{\dom}{dom}
\DeclareMathOperator{\cod}{cod}
\DeclareMathOperator{\id}{id}

\newcommand{\red}[1]{\textcolor{red}{#1}}
\renewcommand{\ker}{\operatorname{Ker}}
\newcommand{\im}{\operatorname{Im}}
\newcommand{\C}{\mathbb{C}}
\newcommand{\R}{\mathbb{R}}
\newcommand{\Q}{\mathbb{Q}}
\newcommand{\N}{\mathbb{N}}
\newcommand{\Z}{\mathbb{Z}}
\newcommand{\D}{\mathbb{D}}
\newcommand{\B}{\mbox{Ob}}
\newcommand{\M}{\mbox{Mo}}
\newcommand{\del}{\Delta}
\newcommand{\odel}[1]{\left[#1\right]}
\newcommand{\Hom}[1]{\text{Hom}(#1)}
\newcommand{\adel}[1]{\left\lbrace #1 \right\rbrace}
%\renewcommand{\theequation}{\thesection.\arabic{equation}}
\newcommand{\funcion}[5]{%
{\setlength{\arraycolsep}{2pt}
\begin{array}{r@{}ccl}
#1\colon \hspace{0pt}& #2 & \longrightarrow & #3\\
& #4 & \longmapsto & #5
\end{array}}}

\newcommand{\func}[3]{#1\colon  #2  \to  #3}

\newcommand\restr[2]{{% we make the whole thing an ordinary symbol
  \left.\kern-\nulldelimiterspace % automatically resize the bar with \right
  #1 % the function
  \vphantom{\big|} % pretend it's a little taller at normal size
  \right|_{#2} % this is the delimiter
  }}

%ENTORNOS

% \theoremstyle{theorem}
\newtheorem{teo}{Theorem}[section]
\newtheorem{prop}[teo]{Proposition}
\newtheorem{lem}[teo]{Lemma}
\newtheorem{cor}[teo]{Corollary}

\theoremstyle{definition}
\newtheorem{defi}[teo]{Definition}
\newtheorem{rem}{Remark}[teo]
\newtheorem{exa}{Example}
\newtheorem{eje}{Exercise}
\newtheorem{que}{Question}

\newenvironment{sol}
  {\begin{proof}[\textit{Solution}]}
  {\end{proof}}

\titlelabel{\thetitle.\quad}

\def\contador{}
\graphicspath{{./figures/}}
\newcommand{\qand}{\quad\text{and}\quad}
\usepackage{microtype,parskip}
\setlength{\parindent}{15pt}
\linespread{1.15}
\usepackage{hyperref}
\hypersetup{
    colorlinks=true,  
    allcolors=blue,
    pdfproducer={Christian Chávez},
}

\makeatletter
\@ifclassloaded{exam}{
    \footer{}{\thepage}{}
    \renewcommand{\thequestion}{\bfseries\arabic{question}}
    \renewcommand{\solutiontitle}{\noindent\textit{Solution.}\enspace}
    \unframedsolutions
}{}
\makeatother


\newenvironment{theproof}
{
    \renewcommand{\solutiontitle}{}
    \begin{solution}
    \vspace*{-\baselineskip}
    \begin{proof}
}
{
    \end{proof}
    \end{solution}
    \renewcommand{\solutiontitle}{\noindent\textit{Solution.} }
}

\usepackage[style=numeric]{biblatex}
\addbibresource{bibliography.bib}

\begin{document}

\def\contador{Lesson 13}
\noindent
\begin{minipage}[c]{0.33\textwidth}
    \includegraphics[scale=0.37]{sello_yachay.png}
\end{minipage}
\begin{minipage}[c]{0.37\textwidth}
    % \centering
    \textbf{\large School of Mathematical and\\ Computational Sciences}\par
    Abstract Algebra
\end{minipage}
~ 
\begin{minipage}[c]{3mm}
    \raggedleft
    \rule[1.5mm]{0.3mm}{15mm}
\end{minipage}
~ 
\begin{minipage}[c]{0.24\textwidth}
    \raggedleft
    Prof. Pablo Rosero\\
    \& Christian Chávez\\
    \contador{}
\end{minipage}

\vspace{1mm}
\noindent\hrulefill

\vspace{3mm}

\section{Ring of Fractions}



Throughout this section, $A \neq \{0\}$ is a commutative ring. The aim of this section is to prove that the commutative ring $A$ is always a subring of a larger ring $Q$ in which every non-zero element that is not a zero divisor is invertible. We had already proved that if $a$ is not a zero divisor nor zero, then $ab = ac$ in $A$ implies that $b = c$. Then a non-zero divisor element has similar properties to invertible elements.

The construction of $\mathbb{Q}$ is usually based on the construction of $\mathbb{Q}$ based on $\mathbb{Z}$. For this, note that every rational number may be represented in many different ways as the quotient of two integers, i.e.,
\[
\frac{a}{b} = \frac{c}{d} \quad \text{iff} \quad ad = bc.
\]
The relation
\[
(a, b) \sim (c, d) \quad \text{iff} \quad ad = bc,
\]
is an equivalence relation over the set $\mathbb{Z} \times \mathbb{Z} \setminus \{0\}$. The fraction $\frac{a}{b}$ is the equivalence class under $\sim$. Then $\mathbb{Q}$ is defined as the set of all equivalence classes, and the operations of addition and multiplication are given by
\[
\frac{a}{b} + \frac{c}{d} = \frac{ad + bc}{bd}, \quad \frac{a}{b} \cdot \frac{c}{d} = \frac{ac}{bd}.
\]

These operations are independent of the choice of representatives of the equivalence classes and make $\mathbb{Q}$ a field. The following theorem is a generalization of the above construction for any commutative ring.

\begin{teo} 
Let $A$ be an integral domain. Then the set $D = A \setminus \{0\}$ is a non-empty subset of $A$ that does not contain any zero divisors, and there is a field $Q$ such that
\begin{enumerate}[label=(\roman*)]
    \item $A$ is embedded in $Q$ as a subring.
    \item Every element of $D$ is invertible in $Q$.
    \item (Uniqueness of $Q$) Up to isomorphism, $Q$ is the smallest field containing $A$ in which all the elements of $D$ are invertible.
\end{enumerate}
\end{teo}

\begin{proof}
    \begin{enumerate}[label=(\roman*)]
        \item Let $T = A \times D$ and define the equivalence relation (this is left as an exercise for the reader) $\sim$ on $T$ by
        \[
        (a, b) \sim (c, d) \iff ad = bc.
        \]
        We denote the equivalence class of $(a, b)$ by $\frac{a}{b}$, i.e.,
        \[
        \frac{a}{b} = \{(c, d) \in T \mid (a, b) \sim (c, d)\}.
        \]
    
        Let's define $Q$ as the set of all the equivalence classes defined above, that is,
        \[
        Q = \left\{\frac{a}{b} \mid (a, b) \in T \right\}.
        \]
        Then $Q$ is a commutative ring with unit with the operations given by
        \[
        \frac{a}{b} + \frac{c}{d} = \frac{ad + bc}{bd}, \quad \frac{a}{b} \cdot \frac{c}{d} = \frac{ac}{bd}.
        \]
    
        Note that the unit of $Q$ is $\frac{e}{e}$, for any $e \in D$, and $\frac{b}{a}$ is the multiplicative inverse of $\frac{a}{b} \in Q$. Since we have already constructed $Q$, we need to embed $A$ into $Q$. For this, we define the ring homomorphism
        \[
        \eta : A \to Q, \quad a \mapsto \frac{ad}{d},
        \]
        where $d$ is any element of $D$ (it's an exercise to prove that $\eta$ is an injective ring homomorphism which is well-defined, i.e., $\eta$ does not depend on the choice of $d \in D$).
    
        \item Let's prove that each $b \in D$ has a multiplicative inverse (under the embedding $\phi$) in $Q$. The element $b$ is represented in $Q$ under $\phi$ by $\frac{bd}{d}$ for any $d \in D$. Then its multiplicative inverse in $Q$ is the fraction $\frac{d}{bd}$. This fact is easy to prove since $A$ is commutative and
        \[
        \frac{bd}{d} \cdot \frac{d}{bd} = \frac{bd \cdot d}{bd^2} = 1 \quad \text{in } Q.
        \]
    
        \item Considering ii), to prove that $Q$ is the smallest ring containing $A$ in which all the elements of $D$ become invertible, is equivalent to proving that, if $R$ is any commutative ring with unit, and
        \[
        \kappa : A \to R
        \]
        is an injective homomorphism such that $\kappa(d)$ is invertible in $R$ for any $d \in D$, then there is an injective homomorphism $\theta : Q \to R$ such that $\theta \circ \eta = \kappa$.
    
        Let $\kappa : A \to R$ be any injective homomorphism such that $\kappa(d)$ is invertible in $R$ for any $d \in D$. Extend $\kappa$ to the well-defined injective ring homomorphism (this is left as an exercise):
        \[
        \theta : Q \to R, \quad \frac{a}{b} \mapsto \kappa(a)(\kappa(b))^{-1}.
        \]
        Then for any $a \in A$ and any $e \in D$:
        \[
        \theta \circ \eta(a) = \theta(\eta(a)),
        \]
        \[
        = \theta\left(\frac{ae}{e}\right),
        \]
        \[
        = \kappa(ae)(\kappa(e))^{-1},
        \]
        \[
        = \kappa(a)\kappa(e)(\kappa(e))^{-1},
        \]
        \[
        = \kappa(a).
        \]
    
        Therefore, $\theta \circ \eta = \kappa$, completing the proof.
    \end{enumerate}
    \end{proof}
    
    The field $Q$ is called the field of fractions of $A$.
    
    \begin{exa}[9.4]
    If $A$ is a field, then its field of fractions is just $A$ itself.
    \end{exa}
    
    \begin{exa}[9.5]
    $\mathbb{Z}$ is an integral domain whose field of fractions is $\mathbb{Q}$. The subring $2\mathbb{Z}$ of $\mathbb{Z}$ has no zero divisors but has no unit, and its field of fractions is also $\mathbb{Q}$.
    \end{exa}

\vfill\red{\fbox{This lecture needs to be reviewed}}

\end{document}