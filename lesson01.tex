\documentclass[11pt,a5paper]{article}

% PAQUETES

\usepackage[spanish]{babel}
\usepackage{amsmath}
\usepackage{amsthm}
\usepackage{amsfonts}
\usepackage[left=1.54cm,right=1.54cm,top=1.54cm,bottom=1.54cm]{geometry}
\usepackage{xfrac}  
\usepackage{tikz-cd}
\usepackage{enumerate}
\usepackage{amsfonts}
\usepackage{amssymb}
\usepackage{tcolorbox}
\usepackage{rotating}
\usepackage{mathpazo}
\usepackage{charter}
\usetikzlibrary{babel}
\usepackage{listings}
\usepackage{amssymb}
\usepackage{extarrows}
\usepackage{makeidx}
\usepackage{graphicx}
\usepackage{multirow}
\usepackage{tikz-cd}
\usepackage{tasks}
\usepackage{color}


%OPERADORES

\DeclareMathOperator{\dom}{dom}
\DeclareMathOperator{\cod}{cod}
\DeclareMathOperator{\id}{id}


\newcommand{\C}{\mathbb{C}}
\newcommand{\im}{\text{Im}}
\newcommand{\R}{\mathbb{R}}
\newcommand{\N}{\mathbb{N}}
\newcommand{\Z}{\mathbb{Z}}
\newcommand{\D}{\mathbb{D}}
\newcommand{\B}{\mbox{Ob}}
\newcommand{\M}{\mbox{Mo}}
\newcommand{\del}{\Delta}
\newcommand{\odel}[1]{\left[#1\right]}
\newcommand{\Hom}[1]{\text{Hom}(#1)}
\newcommand{\adel}[1]{\left\lbrace #1 \right\rbrace}
%\renewcommand{\theequation}{\thesection.\arabic{equation}}
\newcommand{\funcion}[5]{%
{\setlength{\arraycolsep}{2pt}
\begin{array}{r@{}ccl}
#1\colon \hspace{0pt}& #2 & \longrightarrow & #3\\
& #4 & \longmapsto & #5
\end{array}}}

\newcommand{\func}[3]{#1\colon  #2  \to  #3}

\newcommand\restr[2]{{% we make the whole thing an ordinary symbol
  \left.\kern-\nulldelimiterspace % automatically resize the bar with \right
  #1 % the function
  \vphantom{\big|} % pretend it's a little taller at normal size
  \right|_{#2} % this is the delimiter
  }}

%ENTORNOS

\theoremstyle{theorem}
\newtheorem{teo}{Theorem}[section]
\newtheorem{prop}{Proposition}[section]
\newtheorem{lem}{Lemma}[section]
\newtheorem{cor}{Corollary}[section]

\theoremstyle{definition}
\newtheorem{defi}{Definition}[section]
\newtheorem{rem}{Remark}
\newtheorem{eje}{Exercise}
\newtheorem{sol}{\textit{Solution}}
\newtheorem{que}{Question}


\def\contador{}
\graphicspath{{./figures/}}


\begin{document}

\def\contador{1}
\begin{minipage}[c]{30mm}
\includegraphics[scale=0.5]{sello_yachay.png}
\end{minipage}
\begin{minipage}[c]{95mm}
\textbf{School of Mathematical and Computational Sciences}
\end{minipage}
\begin{minipage}[c]{2mm}
\rule[2.5mm]{0.3mm}{15mm}
\end{minipage}
\begin{minipage}[c]{50mm}
Prof. Pablo Rosero.\\ Abstract Algebra: Lesson \contador{}
\end{minipage}

\hrulefill
\vspace{3mm}

\section{Basic properties of the integers}
1

In this lesson and onwards, we consider $\Z$ to be the set of integers numbers, whereas $\Z^{+}$ is the set of strictly positive integers numbers.

\begin{defi}
Let $a,b\in \Z$, with $a\neq 0$. We say  $a$ is a divisor of $b$ if there is an integer $c$ such that $a\cdot c=b$. In this case, we write $a\mid b.$ 
\end{defi}

\begin{rem}
    If $a$ does not divide $b$, we write $a\nmid b$.
\end{rem}

\begin{teo}
    Let $a,b\in \Z\setminus\{0\}$, there is a unique positive integer $d$, called the \textbf{greatest common divisor of $a$ and $b$}, satisfying 
    \begin{enumerate}
        \item $d\mid a$ and $d\mid b.$
        \item If $e\mid a$ and $e\mid b$ then $e\mid d.$
    \end{enumerate}
\end{teo}

\begin{rem}
    If $d$ is the greatest common divisor of $a$ and $b$, we write $d=(a,b).$
    In the  particular case when  $(a,b)=1$, we say  $a$ and $b$ are coprimes.
\end{rem}

\begin{que}
    Why does $(a,b)$ always exist for $a,b\in \Z\setminus\{0\}$?
\end{que}


\begin{teo}[Division algorithm]
    If \(a,b\in \Z\setminus\{0\}\), 
    there are unique \(q,r\in Z\)
    such that 
    \[a= qb +r \qand 0\leq r < |b|.\]
    We call \(q\) the quotient and \(r\) the remainder.
\end{teo}

\begin{proof}
    
\end{proof}



\paragraph{Euclidean Algortihm.} This is an efficient method to compute the gcd of any  two integers.
It is based on the division algorithm.
(Keep in mind that, despite the name, the \textit{division algorithm} is a theorem whereas the \textit{euclidean algorithm} is a procedure.)

If \(a\) and \(b\) are nonzero integers, then by the division algorithm we get \(q,r\in \Z\) such that \(a= qb+r\).
Let \(q_0=q\) and \(r_0=r\).
By applying the division algorithm again with \(q_0\) and \(r_0\) we obtain a new quotient \(q_1\)   and a new remainder \(r_1\).
The idea of this procedure is to continue applying the division algorithm until we reach a zero remainder. From one step to the next,  the divisor becomes the dividend and the remainder the divisor, as follows:
\begin{align*}
a & =q_0 b+r_0 \\
b & =q_1 r_0+r_1 \\
r_0 & =q_2 r_1+r_2 \\
r_1 & =q_3 r_2+r_3 \\
&\;\; \vdots \\
r_{n-2} & =q_n r_{n-1}+r_n \\
r_{n-1} & =q_{n+1} r_n
\end{align*}

\begin{rem}
    Keep in mind the condition \(0\leq r < |b|\) in the division algorithm.
    This means the remainder always gets smaller.
\end{rem}

\begin{que}
    Why the Euclidean algorithm always terminates? In other words,
    why we always get a zero remainder at the end of the Euclidean algorithm?
\end{que}

\begin{eje}
    Compute \((1761, 1567)\) and write this integer as a linear combination of \(1761\) and \(1567\).
\end{eje}

\begin{sol} By the Euclidean algorithm,
   \begin{align*}
    1761 &= 1\cdot 1567 + 194\\
    1567 &= 8\cdot194 + 15\\
    194  &= 12\cdot 15 + 14\\
    15   &= 1\cdot 14 + 1 \\
    14   &= 14\cdot 1 + 0.
   \end{align*}
   From the next to laxt line we get \((1761,1567) = 1\).
\end{sol}

\begin{defi}
    An integer \(p\) is prime iff 
    \begin{enumerate}[label=(\roman*)]
        \item \(p>1\), and
        \item the only positive divisors of \(p\) are \(p\) and \(1\).
    \end{enumerate}
    An integer is \textit{composite} iff it not prime.
\end{defi}


\begin{rem}
    If $p$ is a prime and $b \in \mathbb{Z}\setminus \{0\}$ then
\[
(p, b)= \begin{cases}p & \text { if } p \mid b \\ s & \text { else }\end{cases}
\]
Prove this claim.
\end{rem}

\begin{prop}
    Let \(I\subseteq \Z\) be such that 
    \begin{enumerate}[label=(\roman*)]
        \item \(0\in I\),
        \item\label{lab:ii.exercise.ideal.of.Z} if  \(a,b\in I\), then \(a-b\in I\), 
        \item\label{lab:iii.exercise.ideal.of.Z} if \(a\in I\) and \(q\in I\), then  \(aq\subseteq I\).
    \end{enumerate}
    Then, there is some nonnegative integer \(d\in I\) such that  \[I = \left\{ dk: k \in \Z \right\}.\]
\end{prop}

\begin{rem}
    If \(A\subseteq \Z\) and \(n\in \Z\), we denote 
    \(nA = \left\{n a : a\in A \right\}\).
    If \(A=\Z\), then \((n) = n\Z\).
    Thus, this result states that \(I=(d)\) for some \(d\in I\).
\end{rem}

\begin{proof}
   If \(I = \left\{ 0 \right\}\), take \(d=0\).
   Suppose \(I\neq \left\{ 0 \right\}\)  and \(a\in I\). 
   By \ref{lab:ii.exercise.ideal.of.Z}, 
   if \(a\in I\), then \(-a\in I\), so \(I\) contains both positive and negative integers.
   Since \(I\cap \Z^+\neq \varnothing\), the Well Ordering Principle (W.O.P.) implies there is a smallest positive integer in \(I\).
    Take \(d\) as this  integer.
    By \ref{lab:iii.exercise.ideal.of.Z}, we have \((d)\subseteq I\).
    Let's see the other inclusion.
    If \(a\in I\), then by the division algorithm, \(a=qd+r\) for some \(q,r\in Z\) with \(0\leq r < d\).
    By \ref{lab:ii.exercise.ideal.of.Z}, \(r=a-qd\in I\).
    However, \(d\)  is the smallest positive integer contained in \(I\).
    Since \(0\leq r <d\), the only possibility for this inequality to be true is when  \(r=0\).
    Therefore \(a=qd\).
    It follows \(I =  (d)\), and the proof is complete.
\end{proof}


\begin{teo}[Euclid's lemma]
    Let \(a,b\in \Z\).
    If \(p\) is prime and \(p\mid ab\), then \(p\mid a\) or \(p\mid b\).
\end{teo}

\begin{proof}
    
\end{proof}

\begin{eje}
    Let \(a_1 a_2 \cdots a_n\in \Z\).
    Prove, by induction, that if \(p\) is prime and \(p\mid a_1 a_2 \cdots a_n\), then there is \(i\in \{1,\ldots,n\}\) such that \(p\mid a_i\), i.e., \(p\) must divide at least one integer in the product.
\end{eje}

The converse of Euclid's lemma is also true.

\begin{prop}
    Let \(p>1\).
    If 
    \[\forall a,b\in \Z  : p\mid ab\implies p\mid a \text{  or  } p\mid b,\]
    then \(p\) is prime.
\end{prop}

\begin{proof}
    By contradiction.
\end{proof}


\begin{prop}
    Let \(a,b,c\in Z\).
    If 
    \begin{enumerate}[label=(\roman*)]
        \item \((a,c) = 1\), and
        \item \(c\mid ab\)
    \end{enumerate}
    then \(c\mid b\).
\end{prop}


\begin{proof}
    
\end{proof}



\begin{defi}
    Let \(a,b\in \Z\) with \(b\neq 0\). We say \(\frac{a}{b}    \) is in lowest terms if \((a,b) = 1\).
\end{defi}

\begin{lem}
    Every nonzero rational number equals a fraction in lowest terms.
\end{lem}

\begin{proof}
    
\end{proof}

\begin{prop}
    \(\sqrt{2}\) is irrational.
\end{prop}

\begin{proof}
    
\end{proof}


\begin{teo}[Fundamental Theorem of Arithmetic]
    
\end{teo}

The following function  computes the amount of smaller integers that are coprime to a given integer.
\begin{defi}[Euler's totient function \(\varphi\)]
    Define \(\varphi \colon \Z^+\to \Z\) by 
    \[\varphi(n) = |\left\{ a\leq n : (a,n) = 1 \right\}|.\]
\end{defi}

\paragraph{Properties.} 

\end{document}
