\documentclass[11pt,a4paper]{article}

% PAQUETES
\usepackage[T1]{fontenc}%
\usepackage[utf8]{inputenc}%

\usepackage[english]{babel}
\usepackage{amsmath}
\usepackage{amsthm}
\usepackage{amsfonts}
\usepackage[%left=1.54cm,right=1.54cm,top=1.54cm,bottom=1.54cm
    margin=1in, includefoot,
]{geometry}
\usepackage{xfrac}  
\usepackage{tikz-cd}
\usepackage{enumerate}
\usepackage{amsfonts}
\usepackage{amssymb}
\usepackage{tcolorbox}
\usepackage{rotating}
\usepackage{mathpazo}
% \usepackage{charter}
\usetikzlibrary{babel}
\usepackage{listings}
\usepackage{amssymb}
\usepackage{extarrows}
\usepackage{makeidx}
\usepackage{graphicx}
\usepackage{multirow}
\usepackage{tikz-cd}
\usepackage{tasks}
\usepackage{xcolor}
\usepackage{mathrsfs} % 2024-11-04

% Christian
\usepackage{enumitem,etoolbox,titlesec}


%OPERADORES

\DeclareMathOperator{\dom}{dom}
\DeclareMathOperator{\cod}{cod}
\DeclareMathOperator{\id}{id}

\newcommand{\red}[1]{\textcolor{red}{#1}}
\renewcommand{\ker}{\operatorname{Ker}}
\newcommand{\im}{\operatorname{Im}}
\newcommand{\C}{\mathbb{C}}
\newcommand{\R}{\mathbb{R}}
\newcommand{\Q}{\mathbb{Q}}
\newcommand{\N}{\mathbb{N}}
\newcommand{\Z}{\mathbb{Z}}
\newcommand{\D}{\mathbb{D}}
\newcommand{\B}{\mbox{Ob}}
\newcommand{\M}{\mbox{Mo}}
\newcommand{\del}{\Delta}
\newcommand{\odel}[1]{\left[#1\right]}
\newcommand{\Hom}[1]{\text{Hom}(#1)}
\newcommand{\adel}[1]{\left\lbrace #1 \right\rbrace}
%\renewcommand{\theequation}{\thesection.\arabic{equation}}
\newcommand{\funcion}[5]{%
{\setlength{\arraycolsep}{2pt}
\begin{array}{r@{}ccl}
#1\colon \hspace{0pt}& #2 & \longrightarrow & #3\\
& #4 & \longmapsto & #5
\end{array}}}

\newcommand{\func}[3]{#1\colon  #2  \to  #3}

\newcommand\restr[2]{{% we make the whole thing an ordinary symbol
  \left.\kern-\nulldelimiterspace % automatically resize the bar with \right
  #1 % the function
  \vphantom{\big|} % pretend it's a little taller at normal size
  \right|_{#2} % this is the delimiter
  }}

%ENTORNOS

% \theoremstyle{theorem}
\newtheorem{teo}{Theorem}[section]
\newtheorem{prop}[teo]{Proposition}
\newtheorem{lem}[teo]{Lemma}
\newtheorem{cor}[teo]{Corollary}

\theoremstyle{definition}
\newtheorem{defi}[teo]{Definition}
\newtheorem{rem}{Remark}[teo]
\newtheorem{exa}{Example}
\newtheorem{eje}{Exercise}
\newtheorem{que}{Question}

\newenvironment{sol}
  {\begin{proof}[\textit{Solution}]}
  {\end{proof}}

\titlelabel{\thetitle.\quad}

\def\contador{}
\graphicspath{{./figures/}}
\newcommand{\qand}{\quad\text{and}\quad}
\usepackage{microtype,parskip}
\setlength{\parindent}{15pt}
\linespread{1.15}
\usepackage{hyperref}
\hypersetup{
    colorlinks=true,  
    allcolors=blue,
    pdfproducer={Christian Chávez},
}

\makeatletter
\@ifclassloaded{exam}{
    \footer{}{\thepage}{}
    \renewcommand{\thequestion}{\bfseries\arabic{question}}
    \renewcommand{\solutiontitle}{\noindent\textit{Solution.}\enspace}
    \unframedsolutions
}{}
\makeatother


\newenvironment{theproof}
{
    \renewcommand{\solutiontitle}{}
    \begin{solution}
    \vspace*{-\baselineskip}
    \begin{proof}
}
{
    \end{proof}
    \end{solution}
    \renewcommand{\solutiontitle}{\noindent\textit{Solution.} }
}

\usepackage[style=numeric]{biblatex}
\addbibresource{bibliography.bib}

\begin{document}

\def\contador{Lesson 3}
\noindent
\begin{minipage}[c]{0.33\textwidth}
    \includegraphics[scale=0.37]{sello_yachay.png}
\end{minipage}
\begin{minipage}[c]{0.37\textwidth}
    % \centering
    \textbf{\large School of Mathematical and\\ Computational Sciences}\par
    Abstract Algebra
\end{minipage}
~ 
\begin{minipage}[c]{3mm}
    \raggedleft
    \rule[1.5mm]{0.3mm}{15mm}
\end{minipage}
~ 
\begin{minipage}[c]{0.24\textwidth}
    \raggedleft
    Prof. Pablo Rosero\\
    \& Christian Chávez\\
    \contador{}
\end{minipage}

\vspace{1mm}
\noindent\hrulefill

\vspace{3mm}

\section{Introduction to Group Theory}

\begin{defi}
    A \textbf{group} is a pair  \(\left( G, \cdot \right)\) where \(G\) is a set and \(\cdot\) is a binary  operation on \(G\) such that 
    \begin{enumerate}[label=(\roman*)]
        \item for all \(a,b,c\in G\), it holds  \((a\cdot b) \cdot c = a \cdot (b\cdot c) \),
        \item there is \(e\in G\)  for all \(a\in G\) such that \(e\cdot a = a\cdot e = a\), and 
        \item for all \(a\in G\), there is \(a^{-1}\in G\) such that \(a\cdot a^{-1} = a^{-1}\cdot a = 1\).
    \end{enumerate}
\end{defi}


Instead of saying \((G,\cdot)\) is a group, we say \textit{\(G\) is a group under \(\cdot\)}.
If the operation is understood from the context, we simply say \(G\) is a group.

% If \(G\) is finite, we say  \(\left( G, \cdot \right)\)  is a finite group.

We will study various types of groups.
Let's begin with one of the most basic.
An \textbf{Abelian group} is a group \(\left( G, \cdot \right)\) with the aditional property that 
\[a\cdot b = b\cdot a \]
for all \(a,b\in G\).
In this case, we use \(+\) insted of \(\cdot\) for the binary operation of the group, and denote its identity by \(0\).
A finite group is a a group where the underlying set is finite.

\begin{exa}
\begin{enumerate}[label=(\roman*)]
    \item Some Abelian groups: \(\left( \Z, + \right)\), \(\left( \mathbb{Q}, + \right)\), \(\left( \mathbb{R}, + \right)\),   \(\left( \Z/n\Z , +\right)\)
    \item More Abelian groups: \(\left( \mathbb{R}\setminus\left\{ 0 \right\}, \cdot \right)\), \(\left( \mathbb{Q}\setminus\left\{ 0 \right\}, \cdot \right)\), \(\left( (\Z/n\Z)^\times , \cdot\right)\). Why do we  use \(\cdot\) instead of \(+\) in this case?
    
    \item \(\left( (\Z/n\Z)^\times , +\right)\) is not a group.
    \item Let \((A, \bullet)\) and \((B, \circ)\) be groups.
    Then the set 
    \[A\times B = \left\{ (a,b)\mid a\in a, b\in B \right\}\]
    is a group under the operation defined by the rule 
    \[(a,b)\cdot(c,d)  = (a\bullet c, b\circ d).\]
    What is the identity of this group? How are inverses defined?
\end{enumerate}
\end{exa}


\begin{prop}
    Suppose \((G, \cdot)\) is a group.
    Then 
    \begin{enumerate}[label=(\roman*)]
        \item The identity of \(G\) is unique.
        \item For any \(a\in G\), its inverse \(a^{-1}\) is unique.
        \item For any \(a,b\in G\), \quad \((ab)^{-1} = b^{-1}a^{-1}\)
        \item Finite products are well-defined.
    \end{enumerate}
\end{prop}

\begin{proof}
\begin{enumerate}[label=(\roman*)]
    \item If both \(e\) and \(f\) are identities of \(G\), then \(e= ef = f\). The first equality is due to the fact    \(f\) is an identity, the second because  \(e\) is also an identity.
    \item Let \(a\in G\). If both \(b\) and \(c\) are inverses of \(a\), then
        \[b = b e = b (ac) = (ba) c = e c = c. \]
    \item Classwork.
    \item Classwork.
\end{enumerate}
\end{proof}

\paragraph{Notation.} Let \(x\) be any element of a group. The product of \(x\) with itself \(n\) times is denoted \(x^{n}\).
The product of \(x^{-1}\) with itself \(n\) times is denoted \(x^{-n}\).
We define \(x^{0} = e\).
If the group is Abelian, we write \(nx\) instead of \(x^n\) and \(-nx\) instead of \(x^{ {-n}}\).
Also, \(0x = 0\).


\begin{rem}
    For an Abelian group we denote \(e= 0\), and, for a nonAbelian group, \(e=1.\)
    When not mentioned, we denote the identity of a group by \(1\).
\end{rem}


\begin{prop}
    Let $G$ be a group and let $a, b \in G$. The left and right cancellation laws hold in $G$, that is, for any \(u,v\in G\) it holds
    \begin{enumerate}[label=(\roman*)]
        \item if $a u=a v$, then $u=v$, and
        \item if $u b=v b$, then $u=v$.
    \end{enumerate}
\end{prop}

\begin{proof}
    Exercise.
\end{proof}


\begin{defi}
    Let  $G$ be a group and $x \in G$.
    The \textbf{order} of $x$ is
    \[\min\left\{ n\in \Z^+ \mid x^n = 1 \right\},\]
    provided it exists. 
    We denote the order of \(x\) by $|x|$. 
    If no such integer exists (i.e., if no positive power of \(x\) is the identity),   \(|x|=\infty\).
    
\end{defi}


\begin{rem}
    This notation must not be regarded as an absolute value.
    The use of the bars  should cause no problem because it is used in a different context.
\end{rem}


\begin{eje}
    Compute \(|\overline{6}|\) in \( \Z/9\Z \) and \(|\overline{2}|\) in \(\left( \Z/7\Z \right)^\times\).
\end{eje}


\end{document}