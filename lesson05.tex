\documentclass[11pt,a4paper]{article}

% PAQUETES
\usepackage[T1]{fontenc}%
\usepackage[utf8]{inputenc}%

\usepackage[english]{babel}
\usepackage{amsmath}
\usepackage{amsthm}
\usepackage{amsfonts}
\usepackage[%left=1.54cm,right=1.54cm,top=1.54cm,bottom=1.54cm
    margin=1in, includefoot,
]{geometry}
\usepackage{xfrac}  
\usepackage{tikz-cd}
\usepackage{enumerate}
\usepackage{amsfonts}
\usepackage{amssymb}
\usepackage{tcolorbox}
\usepackage{rotating}
\usepackage{mathpazo}
% \usepackage{charter}
\usetikzlibrary{babel}
\usepackage{listings}
\usepackage{amssymb}
\usepackage{extarrows}
\usepackage{makeidx}
\usepackage{graphicx}
\usepackage{multirow}
\usepackage{tikz-cd}
\usepackage{tasks}
\usepackage{xcolor}
\usepackage{mathrsfs} % 2024-11-04

% Christian
\usepackage{enumitem,etoolbox,titlesec}


%OPERADORES

\DeclareMathOperator{\dom}{dom}
\DeclareMathOperator{\cod}{cod}
\DeclareMathOperator{\id}{id}

\newcommand{\red}[1]{\textcolor{red}{#1}}
\renewcommand{\ker}{\operatorname{Ker}}
\newcommand{\im}{\operatorname{Im}}
\newcommand{\C}{\mathbb{C}}
\newcommand{\R}{\mathbb{R}}
\newcommand{\Q}{\mathbb{Q}}
\newcommand{\N}{\mathbb{N}}
\newcommand{\Z}{\mathbb{Z}}
\newcommand{\D}{\mathbb{D}}
\newcommand{\B}{\mbox{Ob}}
\newcommand{\M}{\mbox{Mo}}
\newcommand{\del}{\Delta}
\newcommand{\odel}[1]{\left[#1\right]}
\newcommand{\Hom}[1]{\text{Hom}(#1)}
\newcommand{\adel}[1]{\left\lbrace #1 \right\rbrace}
%\renewcommand{\theequation}{\thesection.\arabic{equation}}
\newcommand{\funcion}[5]{%
{\setlength{\arraycolsep}{2pt}
\begin{array}{r@{}ccl}
#1\colon \hspace{0pt}& #2 & \longrightarrow & #3\\
& #4 & \longmapsto & #5
\end{array}}}

\newcommand{\func}[3]{#1\colon  #2  \to  #3}

\newcommand\restr[2]{{% we make the whole thing an ordinary symbol
  \left.\kern-\nulldelimiterspace % automatically resize the bar with \right
  #1 % the function
  \vphantom{\big|} % pretend it's a little taller at normal size
  \right|_{#2} % this is the delimiter
  }}

%ENTORNOS

% \theoremstyle{theorem}
\newtheorem{teo}{Theorem}[section]
\newtheorem{prop}[teo]{Proposition}
\newtheorem{lem}[teo]{Lemma}
\newtheorem{cor}[teo]{Corollary}

\theoremstyle{definition}
\newtheorem{defi}[teo]{Definition}
\newtheorem{rem}{Remark}[teo]
\newtheorem{exa}{Example}
\newtheorem{eje}{Exercise}
\newtheorem{que}{Question}

\newenvironment{sol}
  {\begin{proof}[\textit{Solution}]}
  {\end{proof}}

\titlelabel{\thetitle.\quad}

\def\contador{}
\graphicspath{{./figures/}}
\newcommand{\qand}{\quad\text{and}\quad}
\usepackage{microtype,parskip}
\setlength{\parindent}{15pt}
\linespread{1.15}
\usepackage{hyperref}
\hypersetup{
    colorlinks=true,  
    allcolors=blue,
    pdfproducer={Christian Chávez},
}

\makeatletter
\@ifclassloaded{exam}{
    \footer{}{\thepage}{}
    \renewcommand{\thequestion}{\bfseries\arabic{question}}
    \renewcommand{\solutiontitle}{\noindent\textit{Solution.}\enspace}
    \unframedsolutions
}{}
\makeatother


\newenvironment{theproof}
{
    \renewcommand{\solutiontitle}{}
    \begin{solution}
    \vspace*{-\baselineskip}
    \begin{proof}
}
{
    \end{proof}
    \end{solution}
    \renewcommand{\solutiontitle}{\noindent\textit{Solution.} }
}

\usepackage[style=numeric]{biblatex}
\addbibresource{bibliography.bib}

\begin{document}

\def\contador{Lesson 5}
\noindent
\begin{minipage}[c]{0.33\textwidth}
    \includegraphics[scale=0.37]{sello_yachay.png}
\end{minipage}
\begin{minipage}[c]{0.37\textwidth}
    % \centering
    \textbf{\large School of Mathematical and\\ Computational Sciences}\par
    Abstract Algebra
\end{minipage}
~ 
\begin{minipage}[c]{3mm}
    \raggedleft
    \rule[1.5mm]{0.3mm}{15mm}
\end{minipage}
~ 
\begin{minipage}[c]{0.24\textwidth}
    \raggedleft
    Prof. Pablo Rosero\\
    \& Christian Chávez\\
    \contador{}
\end{minipage}

\vspace{1mm}
\noindent\hrulefill

\vspace{3mm}

\section{Homomorphisms and isomorphisms}

When studying an algebraic structure, we are not only interested in the objects that possess this structure, but also in the morphisms between objects of the same type that preserve such structure. 
This is true through out all branches of mathematics.

In the case of group theory, the objects are groups and  the structure preserving maps are called group-homomorphisms.

Let \((G,\cdot)\) and \((H, *) \) be groups.

\begin{defi}
    A group-homomorphism from \(G\) to \(H\) is a map \(\varphi \colon G\to H\) such that 
    \[\varphi(a\cdot b) = \varphi(a) * \varphi(b)\]
    for all \(a,b\in G\).
\end{defi}

\begin{rem}
    From now and on, we omit the adjective ``group'' in ``group-homomorphism'', since we will be working only on the realm of group theory \((\mathbf{Grp})\).
\end{rem}

Note that \(\varphi\) transforms a product \(a\cdot b\) (using the operation of \(G\))  into the   product \(\varphi(a)*\varphi(b)\) (using the operation of \(H\)).
This is why we say \(\varphi\) preserves the structure: it takes a product in \(G\) and maps it to a product in \(H\).
In this case (of groups) there are no more structure to be preserved.

However, the definition above just guarantees that the structure is preserved   in one direction only: from \(G\) to \(H\).
If, in addition, it is possible to preserve 
the structure the other way around, we obtain a more interesting type of map: an \textit{isomorphism}.

\begin{defi}
    An \textbf{isomorphism} is a bijective homomorphism.
\end{defi}

Thus, an isomorphism \(\varphi\colon G\to H\) is just a homomorphism that has   inverse \(\varphi^{-1}\colon H\to G\).
In this case, \(G\) and \(H\) are said to be of the same isomorphism type, or \textit{isomorphic} for short. We write \(G\cong H\).
You may ask \textit{isn't  it necessary that  the inverse   be a homomorphism also (so that the structure is preserved the other way around)?}
And it turns out that \textit{no}, because it follows straight from the definition. (Why?)

\begin{eje}
    Prove \(\varphi^{-1}\) is a homomorphism if \(\varphi\) is an isomorphism.
\end{eje}

\begin{eje}
    Prove \(\cong\) is an equivalence relation (over which set?)
\end{eje}


\begin{lem}\label{lem:facts.pf.group.isomorphism}
    If \(\varphi\colon G\to H\) is an isomorphism, then 
    \begin{enumerate}[label=(\roman*)]
        \item \(|G| = |H|\)
        \item \(G\) is Abelian if and only if \(H\) is Abelian
        \item \(\varphi\) preserves the order of elements, that is, \(|x| = |\varphi(x)|\).
    \end{enumerate}
\end{lem}

\begin{exa}
\begin{enumerate}[label=(\roman*)]
    \item Let us prove \(\left( \R, +  \right)\cong \left( \R^+, \cdot \right)\).
    Define \(\psi\colon \R\to \R^+:x\mapsto \exp(x)\). 
    Notice \[\psi(x + y) = \exp(x+y) = \exp(x)\cdot\exp(y) = \psi(x)\cdot\psi(y).\]
    Since \(\psi\) is injective and surjective (facts known from elementary calculus), \(\psi\) is an isomorphism.
    (Are there any other maps that show \(\left( \R, +  \right)\cong \left( \R^+, \cdot \right)\)?)
    \item If \(X\) and \(Y\) have the same cardinality, then \(S_X\cong S_Y\).
    Indeed, if \(\omega \colon X\to Y\) is a bijection (recall the definition of cardinality), and
    given any \(\alpha\in S_X\), 
    the map \(\omega \circ \alpha\circ \omega^{-1}\colon Y\to Y \) is also a bijection.
    Thus, define \(\varphi\colon S_X\to S_Y\) by \(\varphi(\alpha) = \omega \circ \alpha\circ \omega^{-1}\) for every \(\alpha\in S_X\).
    Diagramatically, we have
    \[\begin{tikzcd}
X \arrow[r, "\alpha", two heads, hook] & X \arrow[r, "\omega", two heads, hook] & Y \arrow[r, "\omega\circ \alpha\circ \omega^{-1}", two heads, hook] &[3em] Y
\end{tikzcd}.\]
    Since \[\varphi(\alpha\circ \beta) = \omega \circ (\alpha\circ\beta)\circ \omega^{-1}= (\omega \circ \alpha\circ \omega^{-1})\circ(\omega\circ\beta\circ \omega^{-1}) = \varphi(\alpha)\circ \varphi(\beta),\]
    \(\varphi\) is a homomorphism.
    The reader must easily verify that \(\varphi\) is an isomorphism.

    \item The groups \(S_3\) and \(\Z/6\Z\) are not isomorphic as one is Abelian and the other is not. This follows from Theorem \ref{lem:facts.pf.group.isomorphism}.
    
    \item \(\left( \R\setminus \left\{ 0 \right\}, \cdot \right) \) and \(\left( \R,+ \right)\) are not isomorphic.
    The former  has an element of  order \(2\) and the latter does not have any element of order \(2\).
\end{enumerate}
\end{exa}

\begin{eje}
\begin{enumerate}[label=(\roman*)]
    \item Provide an example of a group with only one element.
    \item Prove \(S_3\) is the unique nonabelian group of order \(6\) up to isomorphism.
    \item Prove \(A\times B\cong B\times A\) if \(A\) and \(B\) are any groups.
\end{enumerate}
\end{eje}



\section{Subgroups}


The objective is not to learn this concepts only, but mainly to learn to use them to study the principal objects of group theory.


\end{document}