\documentclass[11pt,a4paper]{article}

% PAQUETES
\usepackage[T1]{fontenc}%
\usepackage[utf8]{inputenc}%

\usepackage[english]{babel}
\usepackage{amsmath}
\usepackage{amsthm}
\usepackage{amsfonts}
\usepackage[%left=1.54cm,right=1.54cm,top=1.54cm,bottom=1.54cm
    margin=1in, includefoot,
]{geometry}
\usepackage{xfrac}  
\usepackage{tikz-cd}
\usepackage{enumerate}
\usepackage{amsfonts}
\usepackage{amssymb}
\usepackage{tcolorbox}
\usepackage{rotating}
\usepackage{mathpazo}
% \usepackage{charter}
\usetikzlibrary{babel}
\usepackage{listings}
\usepackage{amssymb}
\usepackage{extarrows}
\usepackage{makeidx}
\usepackage{graphicx}
\usepackage{multirow}
\usepackage{tikz-cd}
\usepackage{tasks}
\usepackage{xcolor}
\usepackage{mathrsfs} % 2024-11-04

% Christian
\usepackage{enumitem,etoolbox,titlesec}


%OPERADORES

\DeclareMathOperator{\dom}{dom}
\DeclareMathOperator{\cod}{cod}
\DeclareMathOperator{\id}{id}

\newcommand{\red}[1]{\textcolor{red}{#1}}
\renewcommand{\ker}{\operatorname{Ker}}
\newcommand{\im}{\operatorname{Im}}
\newcommand{\C}{\mathbb{C}}
\newcommand{\R}{\mathbb{R}}
\newcommand{\Q}{\mathbb{Q}}
\newcommand{\N}{\mathbb{N}}
\newcommand{\Z}{\mathbb{Z}}
\newcommand{\D}{\mathbb{D}}
\newcommand{\B}{\mbox{Ob}}
\newcommand{\M}{\mbox{Mo}}
\newcommand{\del}{\Delta}
\newcommand{\odel}[1]{\left[#1\right]}
\newcommand{\Hom}[1]{\text{Hom}(#1)}
\newcommand{\adel}[1]{\left\lbrace #1 \right\rbrace}
%\renewcommand{\theequation}{\thesection.\arabic{equation}}
\newcommand{\funcion}[5]{%
{\setlength{\arraycolsep}{2pt}
\begin{array}{r@{}ccl}
#1\colon \hspace{0pt}& #2 & \longrightarrow & #3\\
& #4 & \longmapsto & #5
\end{array}}}

\newcommand{\func}[3]{#1\colon  #2  \to  #3}

\newcommand\restr[2]{{% we make the whole thing an ordinary symbol
  \left.\kern-\nulldelimiterspace % automatically resize the bar with \right
  #1 % the function
  \vphantom{\big|} % pretend it's a little taller at normal size
  \right|_{#2} % this is the delimiter
  }}

%ENTORNOS

% \theoremstyle{theorem}
\newtheorem{teo}{Theorem}[section]
\newtheorem{prop}[teo]{Proposition}
\newtheorem{lem}[teo]{Lemma}
\newtheorem{cor}[teo]{Corollary}

\theoremstyle{definition}
\newtheorem{defi}[teo]{Definition}
\newtheorem{rem}{Remark}[teo]
\newtheorem{exa}{Example}
\newtheorem{eje}{Exercise}
\newtheorem{que}{Question}

\newenvironment{sol}
  {\begin{proof}[\textit{Solution}]}
  {\end{proof}}

\titlelabel{\thetitle.\quad}

\def\contador{}
\graphicspath{{./figures/}}
\newcommand{\qand}{\quad\text{and}\quad}
\usepackage{microtype,parskip}
\setlength{\parindent}{15pt}
\linespread{1.15}
\usepackage{hyperref}
\hypersetup{
    colorlinks=true,  
    allcolors=blue,
    pdfproducer={Christian Chávez},
}

\makeatletter
\@ifclassloaded{exam}{
    \footer{}{\thepage}{}
    \renewcommand{\thequestion}{\bfseries\arabic{question}}
    \renewcommand{\solutiontitle}{\noindent\textit{Solution.}\enspace}
    \unframedsolutions
}{}
\makeatother


\newenvironment{theproof}
{
    \renewcommand{\solutiontitle}{}
    \begin{solution}
    \vspace*{-\baselineskip}
    \begin{proof}
}
{
    \end{proof}
    \end{solution}
    \renewcommand{\solutiontitle}{\noindent\textit{Solution.} }
}

\usepackage[style=numeric]{biblatex}
\addbibresource{bibliography.bib}

\begin{document}

\def\contador{Lesson 6}
\noindent
\begin{minipage}[c]{0.33\textwidth}
    \includegraphics[scale=0.37]{sello_yachay.png}
\end{minipage}
\begin{minipage}[c]{0.37\textwidth}
    % \centering
    \textbf{\large School of Mathematical and\\ Computational Sciences}\par
    Abstract Algebra
\end{minipage}
~ 
\begin{minipage}[c]{3mm}
    \raggedleft
    \rule[1.5mm]{0.3mm}{15mm}
\end{minipage}
~ 
\begin{minipage}[c]{0.24\textwidth}
    \raggedleft
    Prof. Pablo Rosero\\
    \& Christian Chávez\\
    \contador{}
\end{minipage}

\vspace{1mm}
\noindent\hrulefill

\vspace{3mm}

\section{Quotient Groups and Homomorphisms}

\subsection{Cosets and counting}

Let \((G,\cdot)\) be a group.


\begin{defi}
Let \(H\leq G\) and \(a,b\in G\).
Define \(\cong_l\) over \(G\) by 
\[a\cong_l b\iff a^{-1}b\in H.\]
Whenever \(a\cong_l b\) we say \(a\) is left congruent to \(b\) module \(H\).
Define \(\cong_r\) over \(G\) by 
\[a\cong_r b\iff ab^{-1}\in H.\]
Whenever \(a\cong_r b\) we say \(a\) is right congruent to \(b\) module \(H\).
\end{defi}

\begin{rem}
    If \(G\) is Abelian, \[a - b\in H\iff b-a\in H\]
    for any \(a,b\in G\).
    This is not true in general unless \(G\) is Abelian.
\end{rem}


\begin{teo}
    Let \(H\leq G\).
    \begin{enumerate}[label=(\roman*)]
        \item Both \(\cong_l\) and \(\cong_r\) are equivalence relations on $G$.
        
        \item The equivalence class of $a \in G$ under left congruence mod $H$ is the set
        \[
         aH = \{ah \mid h \in H\}
        \]
    \item The equivalence class of $a \in G$ under right  congruence mod $H$ is the set
    \[
    Ha = \{ha \mid h \in H\}  .
    \]


    \item For any \(a\in G\), \(\left|H a\right|=|H|=|a H|\).
    \end{enumerate}
\end{teo}

\begin{proof}
    Try it yourself (or classwork).
\end{proof}


We call \(aH\) the  \textit{left coset} of \(H\) by \(a\) in \(G\), and \(Ha\) the a \textit{right coset} of \(H\) by \(a\) in \(G\). 

\begin{rem}
    In additive notation (that is, when we are working with an Abelian group) we write \(a+H\) instead of \(aH\) and \(H+a\) instead of \(Ha\).
    In fact, there is no difference between left and right cosets in this case. (Why \(a+ H = H + a\) for any \(a\in G\)?)
\end{rem}
 



\begin{cor}\label{cor.hungerford.cor4.3}
Let \(H\leq G\).
\begin{enumerate}[label=(\roman*)]
    \item      \( G = \bigcup\limits_{a \in G} Ha = \bigcup\limits_{a \in G} aH \)
    \item For all \(a,b\in G\) distinct, \( aH \cap bH = \emptyset \) and \( Ha \cap Hb = \emptyset \).
    \item For all \(a,b\in G\), we have \( aH = bH \) if and only if \( a^{-1}b \in H \) (or \( b-a \in H \) in additive notation)  and \( Ha = Hb \) if and only if \( ab^{-1} \in H \) (or \( b-a  \in H \) in additive notation).
    \item If \( \mathcal{R} = \{ Ha \mid a \in G \} \) and \( \mathcal{L} = \{ aH \mid a \in G \} \) then \(|\mathcal{R}|=|\mathcal{L}|\).
\end{enumerate}
 
\end{cor}

A special name and notation have been adopted for the number of left (or right) cosets of a subgroup in a group.

\begin{defi}[Index]
    The index of a subgroup $H$ in $G$ is the number of distinct left cosets of $H$ in $G$. This number is denoted by $|G: H|$. 
\end{defi}

\begin{eje}
    Prove $|G: H|$ equals the number of distinct right cosets of $H$ in $G$.
    Thus it does not matter whether we count left or right cosets.
\end{eje}


\begin{defi}
    A \textbf{complete set of right representatives} of \(H\) is a subset \(S\) of \(G\) consisting of exactly one element from each right coset. In other words, 
    \(S\cap Ha \) is a singleton for every \(a\in G\).
\end{defi}

Define a \textit{complete set of left representatives} in the obvious way.
Note that such a set (either left or right) contains exactly one element of \(H\) since \(H = He\), where \(e\) is the identity of \(G\).
(What is the cardinality of a complete set of representatives?)
Further, if \(H=\langle e\rangle\), then \(Ha = \{a\} \) for any \(a\in G\), and \(|G:H| = |H|\), that is, there are as many left (or right) cosets as the number of elements of \(G\).

\begin{teo}
    If $K$, $H$, $G$ are groups  with $K<H<G$ then
\[
|G: K|=|G: H| \cdot|H: K|
\]
If any two of these indices are finite, so is the third.
\end{teo}

\begin{proof}
    Let \(\Lambda\) be a complete set of right representatives of \(H\) in \(G\).
    By Corollary \ref{cor.hungerford.cor4.3}, \[G=\bigcup_{a \in G} Ha =  \bigcup_{a \in \Lambda} Ha.\]
    Basically, we are joining all the equivalence classes given by right congruence modulo \(H\), and their union covers \(G\) (why?)
    Similarly, let \(\Omega\) be a complete set of right representatives of \(K\) in \(H\) and write \[H = \bigcup_{b \in \Omega } Kb  .\]
    Therefore, \[G = \bigcup_{a \in\Lambda} Ha =\bigcup_{a \in\Lambda} \left( \bigcup_{b \in \Omega } Kb \right)a = \bigcup_{(a,b) \in \Lambda\times \Omega} Kba\]
    (If you are not confortable with this, you should review the definition of a union over a multiple indexed family of sets. See \href{https://en.wikipedia.org/wiki/List_of_set_identities_and_relations#Definitions}{here} and \href{https://www.gornahoor.net/library/Halmos_NaiveSetTheory.pdf}{here} (page 35).)
    Let's now prove that the cosets \(Kba\) are mutually disjoint.
    Suppose \(Kba = Kb'a'\).
    Then \(ba = kb'a'\) for some \(k\in K\).
    Since \(b,b',k\in H\), have \(Ha = H ba = H kb'a' = H a'\), whence \(a=a'\) because we are working with complete sets of representatives.
    Thus \(b = kb'\).
    The same reasoning gives \(Kb = Kkb' = Kb'\) whence \(b=b'\).
    This proves the cosets \(Kba\) are pairwise disjoint.
    Finally, it follows that \(|G:K| = |\Lambda\times \Omega|\) by definition of index, and so  \[|G:K| = |\Lambda||\Omega| = |G:H| |H:K|,\]
    as desired.
    The last statement of the theorem is obvious.
\end{proof}


\begin{cor}[Lagrange's theorem]
    If \( H \leq G \), then \( |G| = [G : H] |H| \). 
In particular, if \( G \) is finite, the order of  any \( a \in G \) divides \( |G| \).
\end{cor}


\begin{proof}
    Apply the last theorem with $K=\langle e\rangle$ for the first statement. The second is a special case of the first with $H=\langle a\rangle$.
\end{proof}


If $G$ is a group and $H, K$ are subsets of $G$, we denote by $H K$ the set $\{a b \mid a \in H, b \in K\}$. Note that a right or left coset of a subgroup is a special case of this construction.

\begin{rem}
    Careful! If $H, K$ are subgroups, $H K$ may not be a subgroup.
\end{rem}

\begin{teo}
    Let \(H\) and \(K\) be finite subgroups of a group \(G\). Then \[|HK| =  \frac{|H||K|}{|H\cap K|}.\]
\end{teo}

\begin{proof}
    Note $C=H \cap K$ is a subgroup of $K$ of index $n=$ $|K| /|H \cap K|$ and $K$ is the disjoint union of right cosets $C k_1 \cup C k_2 \cup \ldots \cup C k_n$ for some $k_i \in K$. Since $H C=H$, this implies that $H K$ is the   union of the disjoint sets $H k_1 \cup H k_2 \cup$ $\cdots \cup H k_n$. Therefore, $|H K|=|H| \cdot n=|H||K| /|H \cap K|$.
\end{proof}


\noindent
\begin{minipage}[c]{0.5\textwidth}
\textbf{Proposition 1.9.} If \( H \) and \( K \) are subgroups of a group \( G \), then \( [H : H \cap K] \leq [G : K] \). If \( [G : K] \) is finite, then \( [H : H \cap K] = [G : K] \) if and only if \( G = HK \).
\end{minipage}%
\hfill
\begin{minipage}[c]{0.45\textwidth}
\[\begin{tikzcd}
    & G \arrow[ld, no head] \arrow[rd, no head] &                       \\
H \arrow[rd, no head] &                                           & K \arrow[ld, no head] \\
    & H\cap G                                   &                      
\end{tikzcd}\]
\end{minipage}

\begin{proof}
    Let $A$ be the set of all right cosets of $H \cap K$ in $H$ and $B$ the set of all right cosets of $K$ in $G$.
    The map $\varphi: A \rightarrow B$ given by $(H \cap K) h \mapsto K h$, with $h \in H$, is well defined since $(H \cap K) h^{\prime}=(H \cap K) h$ implies $h^{\prime} h^{-1} \in H \cap K \subset K$ and hence $K h^{\prime}=K h$.
    Complete the proof by following these simple steps:
    \begin{enumerate}[label=(\roman*)]
        \item Show that $\varphi$ is injective. Then $|H: H \cap K|=|A| \leq|B|$ $=|G: K|$.
        \item If $|G: K|$ is finite, then show that $|H: H \cap K|=|G: K|$ if and only if $\varphi$ is surjective.
        \item $\varphi$ is surjective if and only if $G=K H$.
    \end{enumerate}
    
    (Hint: note that for $h \in H$ and $k \in K$, we have $ K k h=K h$ since $(k h) h^{-1}=k \in K$.) Conclude.
\end{proof}


\begin{prop}
    Let \( H \) and \( K \) be subgroups of \( G \) with finite index of a group \( G \). Then \( |G : H \cap K| \) is finite and \(|G : H \cap K| \leq |G : H| |G : K| \). Furthermore, \(|G : H \cap K| = |G : H| |G : K| \) if \( G = HK \).
\end{prop}

\begin{proof}
    Classwork.
\end{proof}



\end{document}