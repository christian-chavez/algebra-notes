\documentclass[11pt,a4paper]{article}

% PAQUETES
\usepackage[T1]{fontenc}%
\usepackage[utf8]{inputenc}%

\usepackage[english]{babel}
\usepackage{amsmath}
\usepackage{amsthm}
\usepackage{amsfonts}
\usepackage[%left=1.54cm,right=1.54cm,top=1.54cm,bottom=1.54cm
    margin=1in, includefoot,
]{geometry}
\usepackage{xfrac}  
\usepackage{tikz-cd}
\usepackage{enumerate}
\usepackage{amsfonts}
\usepackage{amssymb}
\usepackage{tcolorbox}
\usepackage{rotating}
\usepackage{mathpazo}
% \usepackage{charter}
\usetikzlibrary{babel}
\usepackage{listings}
\usepackage{amssymb}
\usepackage{extarrows}
\usepackage{makeidx}
\usepackage{graphicx}
\usepackage{multirow}
\usepackage{tikz-cd}
\usepackage{tasks}
\usepackage{xcolor}
\usepackage{mathrsfs} % 2024-11-04

% Christian
\usepackage{enumitem,etoolbox,titlesec}


%OPERADORES

\DeclareMathOperator{\dom}{dom}
\DeclareMathOperator{\cod}{cod}
\DeclareMathOperator{\id}{id}

\newcommand{\red}[1]{\textcolor{red}{#1}}
\renewcommand{\ker}{\operatorname{Ker}}
\newcommand{\im}{\operatorname{Im}}
\newcommand{\C}{\mathbb{C}}
\newcommand{\R}{\mathbb{R}}
\newcommand{\Q}{\mathbb{Q}}
\newcommand{\N}{\mathbb{N}}
\newcommand{\Z}{\mathbb{Z}}
\newcommand{\D}{\mathbb{D}}
\newcommand{\B}{\mbox{Ob}}
\newcommand{\M}{\mbox{Mo}}
\newcommand{\del}{\Delta}
\newcommand{\odel}[1]{\left[#1\right]}
\newcommand{\Hom}[1]{\text{Hom}(#1)}
\newcommand{\adel}[1]{\left\lbrace #1 \right\rbrace}
%\renewcommand{\theequation}{\thesection.\arabic{equation}}
\newcommand{\funcion}[5]{%
{\setlength{\arraycolsep}{2pt}
\begin{array}{r@{}ccl}
#1\colon \hspace{0pt}& #2 & \longrightarrow & #3\\
& #4 & \longmapsto & #5
\end{array}}}

\newcommand{\func}[3]{#1\colon  #2  \to  #3}

\newcommand\restr[2]{{% we make the whole thing an ordinary symbol
  \left.\kern-\nulldelimiterspace % automatically resize the bar with \right
  #1 % the function
  \vphantom{\big|} % pretend it's a little taller at normal size
  \right|_{#2} % this is the delimiter
  }}

%ENTORNOS

% \theoremstyle{theorem}
\newtheorem{teo}{Theorem}[section]
\newtheorem{prop}[teo]{Proposition}
\newtheorem{lem}[teo]{Lemma}
\newtheorem{cor}[teo]{Corollary}

\theoremstyle{definition}
\newtheorem{defi}[teo]{Definition}
\newtheorem{rem}{Remark}[teo]
\newtheorem{exa}{Example}
\newtheorem{eje}{Exercise}
\newtheorem{que}{Question}

\newenvironment{sol}
  {\begin{proof}[\textit{Solution}]}
  {\end{proof}}

\titlelabel{\thetitle.\quad}

\def\contador{}
\graphicspath{{./figures/}}
\newcommand{\qand}{\quad\text{and}\quad}
\usepackage{microtype,parskip}
\setlength{\parindent}{15pt}
\linespread{1.15}
\usepackage{hyperref}
\hypersetup{
    colorlinks=true,  
    allcolors=blue,
    pdfproducer={Christian Chávez},
}

\makeatletter
\@ifclassloaded{exam}{
    \footer{}{\thepage}{}
    \renewcommand{\thequestion}{\bfseries\arabic{question}}
    \renewcommand{\solutiontitle}{\noindent\textit{Solution.}\enspace}
    \unframedsolutions
}{}
\makeatother


\newenvironment{theproof}
{
    \renewcommand{\solutiontitle}{}
    \begin{solution}
    \vspace*{-\baselineskip}
    \begin{proof}
}
{
    \end{proof}
    \end{solution}
    \renewcommand{\solutiontitle}{\noindent\textit{Solution.} }
}

\usepackage[style=numeric]{biblatex}
\addbibresource{bibliography.bib}

\begin{document}

\def\contador{Lesson 7}
\noindent
\begin{minipage}[c]{0.33\textwidth}
    \includegraphics[scale=0.37]{sello_yachay.png}
\end{minipage}
\begin{minipage}[c]{0.37\textwidth}
    % \centering
    \textbf{\large School of Mathematical and\\ Computational Sciences}\par
    Abstract Algebra
\end{minipage}
~ 
\begin{minipage}[c]{3mm}
    \raggedleft
    \rule[1.5mm]{0.3mm}{15mm}
\end{minipage}
~ 
\begin{minipage}[c]{0.24\textwidth}
    \raggedleft
    Prof. Pablo Rosero\\
    \& Christian Chávez\\
    \contador{}
\end{minipage}

\vspace{1mm}
\noindent\hrulefill

\vspace{3mm}

\section{Cyclic groups and cyclic subgroups}

\begin{defi}
    A group $H$ is cyclic if $H$ can be generated by a single element, i.e., there exists $a \in H$ such that
\[
H =  \{a^n \mid n \in \mathbb{Z}\} .
\]
In this case, we denote \(H =   \langle a \rangle\).
\end{defi}

\begin{exa}
    In additive notation $\mathbb{Z}/n\mathbb{Z}$ is cyclic and $\mathbb{Z}/n\mathbb{Z} = \langle \,\overline{1} \,\rangle$
\end{exa}


\begin{rem}\hfill\null
    \begin{enumerate}[label=(\roman*)]
        \item In additive notation $H = \{na \mid n \in \mathbb{Z}\}$. 
        \item If $ H  = \langle x \rangle$ then $ H  = \langle x^{-1} \rangle$ also. This means $x$, the generator, is not unique.
        \item We could have  $x^n =  x^m$ even if $n \neq m$. For instance, in the example above, \(2\cdot \overline{1} = (n+2)\cdot \overline{1}\)
        \item      Every cyclic subgroup $H$ is Abelian. For example, if $H = \langle r \rangle$ in $G = D_n$, then $H$ is Abelian, but $D_n$ is not cyclic.
        \item By convention, \(x^0 = 1\) for any element \(x\).
    \end{enumerate}
\end{rem}

\begin{eje}
    If $G = D_{2n}$ and \(H\leq G\) the subgroup consisting of rotations, then $H = \langle r \rangle$ and $r^k = r^m$ if and only if $k \equiv m (\bmod n)$.
\end{eje}


\begin{prop}
    If \( H = \langle x \rangle \) then \( \lvert H \rvert = \lvert x \rvert \). More specifically:
    \begin{enumerate}[label=(\roman*)]
        \item If \( \lvert H \rvert = n < \infty \), then \( x^n = 1 \) and \( 1, x, \dots, x^{n-1} \) are all distinct elements of \( H \).
        \item If \( \lvert H \rvert = \infty \), then \( x^n \neq 1 \) for \( n \neq 0 \) and \( x^a \neq x^b \) for \( a \neq b \) in \( \mathbb{Z} \).
    \end{enumerate}
\end{prop}


\begin{prop}
    Let \( G \) be a group, \( x \in G \), and \( m, n \in \mathbb{Z} \backslash \{0\} \).
\begin{itemize}
    \item If \( x^m = 1 \) and \( x^n = 1 \), then \( x^d = 1 \) where \( d = (m, n) \).
    \item In particular, if \( x^m = 1 \), then \( x^{\lvert m \rvert} = 1 \).
\end{itemize}
\end{prop}



\begin{proof}
    There exist \( r, s \in \mathbb{Z} \) such that \( d = mr + ns \) where \( d = (m, n) \).
    Therefore, \[ x^d = (x^m)^r \cdot (x^n)^s = 1^r \cdot 1^s = 1 .\]
    If \( x^m = 1 \), let \( n = |x| \). If \( m = 0 \), certainly \( n \mid m \), so we may assume \( m \neq 0 \). Since some nonzero power of \( x \) is the identity, \( n < \infty \). Let \( d = (m, n) \) so by the preceding result \( x^d = 1 \). Since \( 0 < d \leq n \) and \( n \) is the smallest positive power of \( x \) which gives the identity, we must have \( d = n \), that is, \( n \mid m \) as asserted.

\end{proof}

 
\begin{teo}
    Any two cyclic groups of the same order are isomorphic.
\end{teo}

\begin{proof}
    
\begin{enumerate}[label=(\roman*)]
    \item[(i)] \textbf{Finite case.}
    Let \( H_1 = \langle x \rangle \) and \( H_2 = \langle y \rangle \) where \( |x| = |y| = n \).
    Define \( \varphi: \langle x \rangle \to \langle y \rangle \) by \( \varphi(x^k) = y^k \).
    Then \( \varphi \) is a well-defined isomorphism. Indeed, If \( x^k = x^l \) then \(x^{k-l} = 1\), whence \(n\mid k-k\). Hence \(nt = k-l\) for some \(t\in\Z\). Thus \(1 = y^{nt} = y^{k-l} \), whence \(y^k=y^l\) and it follows that  \( \varphi(x^k) = \varphi(x^l) \).
    Note \(\varphi\)  is a homomorphism because \[\varphi(x^k \cdot x^l) = \varphi(x^{k+l}) = y^{k+l} = y^k \cdot y^l = \varphi(x^k) \cdot \varphi(x^l).\] 
    Moreover, \(\varphi\) is surjective since if \( y^k \in \langle y \rangle \) then \( \varphi(x^k) = y^k \).
    Recall that every surjective function between finite sets of the same cardinality is bijective (prove this if you have not seen it). 
    \item[(ii)] \textbf{Infinite case.}
    If \( H = \langle x \rangle \) with \( |H| = \infty \), then define \( \varphi\colon  \mathbb{Z} \to \langle x \rangle \) by \( \varphi(k) = x^k \).
    It is clear that 
    \( \varphi \) is an isomorphism. (If it is not clear for you, prove it.)
\end{enumerate}
\end{proof}

\begin{rem}
    The second part of this proof tell us that, up to isomorphism, there exists a unique cyclic group of finite order \( n \), namely \( \mathbb{Z}/n\mathbb{Z}  \), and a unique cyclic group of infinite order, namely \( \mathbb{Z} \).
\end{rem}


\begin{prop}
    Let \( G \) be a group, let \( x \in G \), and let \( a \in \mathbb{Z}\setminus\{0\} \).
\begin{enumerate}[label=(\roman*)]
    \item If \( |x| = \infty \), then \( |x^a| = \infty \).
    \item If \( |x| = n < \infty \), then \( |x^a| = \frac{n}{(n, a)} \).
    \item If \( |x| = n < \infty \) and   \( a > 0  \) is such that \(a\mid n\), then \( |x^a| = \frac{n}{a} \).
\end{enumerate}
\end{prop}


\begin{proof}
    \begin{enumerate}[label=(\roman*)]
        \item By way of contradiction assume \( |x| = \infty \) but \( |x^a| = m < \infty \). By definition of order
        \[
        1 = (x^a)^m = x^{am}.
        \]
        Also,
        \[
        x^{-am} = (x^am)^{-1} = 1^{-1} = 1.
        \]
        Now one of \( am \) or \( -am \) is positive (since neither \( a \) nor \( m \) is \( 0 \)) so some positive power of \( x \) is the identity. This contradicts the hypothesis \( |x| = \infty \), so the assumption \( |x^a| < \infty \) must be false. The result is is established.
    
        \item Let
        \[
        y = x^a, \quad (n, a) = d \quad \text{and write} \quad n = db, \quad a = dc,
        \]
        for suitable \( b, c \in \mathbb{Z} \) with \( b > 0 \). Since \( d \) is the greatest common divisor of \( n \) and \( a \), the integers \( b \) and \( c \) are relatively prime,
        \(
        (b, c) = 1.
        \)
        We must show \( |y| = b \). First note that
        \[
        y^b = x^{ab} = x^{dcb} = (x^{dc})^b = (x^n)^c = 1^c = 1
        \]
        so,   we see that \( |y| \) divides \( b \). Let \( k = |y| \). Then
        \[
        x^{ak} = y^k = 1,
        \]
        so   \( n \mid ak \), i.e., \( db \mid dck \). Thus \( b \mid ck \). Since \( b \) and \( c \) have no factors in common, \( b \) must divide \( k \). Since \( b \) and \( k \) are positive integers which divide each other, \( b = k \). 
    
        
        \item This is a special case of the last item.
    \end{enumerate}
\end{proof}

\begin{prop}
    Let \( H = \langle x \rangle \).
    \begin{enumerate}[label=(\roman*)]
        \item Assume \( |x| = \infty \). Then \( H = \langle x^a \rangle \) if and only if \( a = \pm 1 \).
        \item Assume \( |x| = n < \infty \). Then \( H = \langle x^a \rangle \) if and only if \( (a, n) = 1 \). In particular, the number of generators of \( H \) is \( \phi(n) \) (where \( \phi \) is Euler's \(\phi\)-function).
    \end{enumerate}
\end{prop}

\begin{proof}
    We leave (i) as an exercise. In (ii) if \( |x| = n < \infty \), note  \( x^a \) generates a subgroup of \( H \) of order \( |x^a| \). This subgroup equals all of \( H \) if and only if \( |x^a| = |x| \). Thus
\[
|x^a| = |x| \quad \text{if and only if} \quad \frac{n}{(a, n)} = n, \quad \text{i.e. if and only if} \quad (a, n) = 1.
\]
Since \( \phi(n) \) is, by definition, the number of \( a \) in \( \{1, 2, \ldots, n\} \) such that \( (a, n) = 1 \), this is the number of generators of \( H \).

\end{proof}

\begin{teo}[Complete structure of a cyclic group]
    Let \( H = \langle x \rangle \) be a cyclic group.
\begin{enumerate}
    \item Every subgroup of \( H \) is cyclic. More precisely, if \( K \leq H \), then either \( K = \{1\} \) or \( K = \langle x^d \rangle \), where \( d \) is the smallest positive integer such that \( x^d \in K \).
    \item If \( |H| = \infty \), then for any distinct nonnegative integers \( a \) and \( b \), \( x^a \neq x^b \). Furthermore, for every integer \( m \), \( x^m = x^{|m|} \), where \( |m| \) denotes the absolute value of \( m \), so that the nontrivial subgroups of \( H \) correspond bijectively with the integers 1, 2, 3, \ldots.
    \item If \( |H| = n < \infty \), then for each positive integer \( a \) dividing \( n \) there is a unique subgroup of \( H \) of order \( a \). This subgroup is the cyclic group \( \langle x^d \rangle \), where \( d = \frac{n}{a} \). Furthermore, for every integer \( m \), \( x^m = x^{(n, m)} \), so that the subgroups of \( H \) correspond bijectively with the positive divisors of \( n \).
\end{enumerate}
\end{teo}

\begin{proof}
    Classwork.
\end{proof}

\begin{rem}
    In \( \mathbb{Z}/n\mathbb{Z} \),
    \begin{enumerate}[label=(\roman*)]
        \item \( \mathbb{Z}/n\mathbb{Z} = \langle \, \overline{1} \, \rangle = \langle \overline{m} \rangle \) if and only if \( (m, n) = 1 \) for \( m \in \mathbb{Z} \).
        \item \( \langle \overline{s} \rangle \leq \langle (\overline{s}, \overline{m}) \rangle \).
        \item \( \langle \overline{a} \rangle \leq \langle \overline{b} \rangle \) if and only if \( (b, n) \mid (a, n) \) where \( 1 \leq a, b \leq n \).
    \end{enumerate}
\end{rem}

\begin{eje}
    Find \(a\in \Z\) such that  \( \mathbb{Z}/48\mathbb{Z} = \langle\overline{a}\rangle \). Find the order of \(\overline{a}\) and the inclusion between the subgroups of \(\mathbb{Z}/48\mathbb{Z}\).  
    Notice that \(48 = 2^4\cdot3\) and \(\varphi(48)= 16\).
\end{eje}

\end{document}