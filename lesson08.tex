\documentclass[11pt,a4paper]{article}

% PAQUETES
\usepackage[T1]{fontenc}%
\usepackage[utf8]{inputenc}%

\usepackage[english]{babel}
\usepackage{amsmath}
\usepackage{amsthm}
\usepackage{amsfonts}
\usepackage[%left=1.54cm,right=1.54cm,top=1.54cm,bottom=1.54cm
    margin=1in, includefoot,
]{geometry}
\usepackage{xfrac}  
\usepackage{tikz-cd}
\usepackage{enumerate}
\usepackage{amsfonts}
\usepackage{amssymb}
\usepackage{tcolorbox}
\usepackage{rotating}
\usepackage{mathpazo}
% \usepackage{charter}
\usetikzlibrary{babel}
\usepackage{listings}
\usepackage{amssymb}
\usepackage{extarrows}
\usepackage{makeidx}
\usepackage{graphicx}
\usepackage{multirow}
\usepackage{tikz-cd}
\usepackage{tasks}
\usepackage{xcolor}
\usepackage{mathrsfs} % 2024-11-04

% Christian
\usepackage{enumitem,etoolbox,titlesec}


%OPERADORES

\DeclareMathOperator{\dom}{dom}
\DeclareMathOperator{\cod}{cod}
\DeclareMathOperator{\id}{id}

\newcommand{\red}[1]{\textcolor{red}{#1}}
\renewcommand{\ker}{\operatorname{Ker}}
\newcommand{\im}{\operatorname{Im}}
\newcommand{\C}{\mathbb{C}}
\newcommand{\R}{\mathbb{R}}
\newcommand{\Q}{\mathbb{Q}}
\newcommand{\N}{\mathbb{N}}
\newcommand{\Z}{\mathbb{Z}}
\newcommand{\D}{\mathbb{D}}
\newcommand{\B}{\mbox{Ob}}
\newcommand{\M}{\mbox{Mo}}
\newcommand{\del}{\Delta}
\newcommand{\odel}[1]{\left[#1\right]}
\newcommand{\Hom}[1]{\text{Hom}(#1)}
\newcommand{\adel}[1]{\left\lbrace #1 \right\rbrace}
%\renewcommand{\theequation}{\thesection.\arabic{equation}}
\newcommand{\funcion}[5]{%
{\setlength{\arraycolsep}{2pt}
\begin{array}{r@{}ccl}
#1\colon \hspace{0pt}& #2 & \longrightarrow & #3\\
& #4 & \longmapsto & #5
\end{array}}}

\newcommand{\func}[3]{#1\colon  #2  \to  #3}

\newcommand\restr[2]{{% we make the whole thing an ordinary symbol
  \left.\kern-\nulldelimiterspace % automatically resize the bar with \right
  #1 % the function
  \vphantom{\big|} % pretend it's a little taller at normal size
  \right|_{#2} % this is the delimiter
  }}

%ENTORNOS

% \theoremstyle{theorem}
\newtheorem{teo}{Theorem}[section]
\newtheorem{prop}[teo]{Proposition}
\newtheorem{lem}[teo]{Lemma}
\newtheorem{cor}[teo]{Corollary}

\theoremstyle{definition}
\newtheorem{defi}[teo]{Definition}
\newtheorem{rem}{Remark}[teo]
\newtheorem{exa}{Example}
\newtheorem{eje}{Exercise}
\newtheorem{que}{Question}

\newenvironment{sol}
  {\begin{proof}[\textit{Solution}]}
  {\end{proof}}

\titlelabel{\thetitle.\quad}

\def\contador{}
\graphicspath{{./figures/}}
\newcommand{\qand}{\quad\text{and}\quad}
\usepackage{microtype,parskip}
\setlength{\parindent}{15pt}
\linespread{1.15}
\usepackage{hyperref}
\hypersetup{
    colorlinks=true,  
    allcolors=blue,
    pdfproducer={Christian Chávez},
}

\makeatletter
\@ifclassloaded{exam}{
    \footer{}{\thepage}{}
    \renewcommand{\thequestion}{\bfseries\arabic{question}}
    \renewcommand{\solutiontitle}{\noindent\textit{Solution.}\enspace}
    \unframedsolutions
}{}
\makeatother


\newenvironment{theproof}
{
    \renewcommand{\solutiontitle}{}
    \begin{solution}
    \vspace*{-\baselineskip}
    \begin{proof}
}
{
    \end{proof}
    \end{solution}
    \renewcommand{\solutiontitle}{\noindent\textit{Solution.} }
}

\usepackage[style=numeric]{biblatex}
\addbibresource{bibliography.bib}

\begin{document}

\def\contador{Lesson 8}
\noindent
\begin{minipage}[c]{0.33\textwidth}
    \includegraphics[scale=0.37]{sello_yachay.png}
\end{minipage}
\begin{minipage}[c]{0.37\textwidth}
    % \centering
    \textbf{\large School of Mathematical and\\ Computational Sciences}\par
    Abstract Algebra
\end{minipage}
~ 
\begin{minipage}[c]{3mm}
    \raggedleft
    \rule[1.5mm]{0.3mm}{15mm}
\end{minipage}
~ 
\begin{minipage}[c]{0.24\textwidth}
    \raggedleft
    Prof. Pablo Rosero\\
    \& Christian Chávez\\
    \contador{}
\end{minipage}

\vspace{1mm}
\noindent\hrulefill

\vspace{3mm}


\section{Subgroup generated by subsets of a group}

Throughout this lesson, \(G\) denotes a group.

\begin{prop}
    If \(\mathcal{A}\) is any nonempty collection of subgroups of \(G\), then 
    \[\bigcap_{H\in \mathcal{A}} H \leq G.\]
\end{prop}



\begin{defi}
Let \(A\) be any subset of \(G\).
The \textbf{subgroup generated by} \(A\) is 
\[\langle A\rangle := \bigcap_{\stackrel{A\subseteq H }{H\leq G }  } H .\]
\end{defi}

This definition says that \(\langle A\rangle \) is the smallest subgroup of \(G\) that contains \(A\). 
It is clear that   the subgroup generated by a subgroup \(H\) is \(H\) itself. 
What would be the subgroup generated by \(\varnothing\)?


\begin{rem}
\begin{enumerate}[label=(\roman*)]
    \item      If \(A \) is a finite set, say \(A = \{a_1, \ldots, a_n \}\),
    then we simply write \[\langle A\rangle = \langle a_1, \ldots, a_n \rangle .\]

    \item Recall from the previous lesson that \(\langle a\rangle\) denotes the cyclic subgroup generated by \(a\).
    With the definition above, it is easy to see that this is the same as the subgroup generated by \(\{a\}\).
    Thus the notation is unambiguous.

    \item If \(A,B\subseteq G\), then we write  \(\langle A, B \rangle\) to mean  \(\langle A\cup B\rangle \).
    This subgroup is denoted \(A \vee B\). 
\end{enumerate}
\end{rem}
 
\begin{defi}
    Let \(A\subseteq     G\).
    Define 
    \[\overline{A} = \left\{ a_1^{k_1} a_2^{k_2} \cdots a_{n}^{k_n} \mid n\in \Z_{\geq 0}, a_i \in A ,\, k_i = \pm 1 \text{ for all } 0 \leq i\leq n \right\}\]
\end{defi}

Note that \(n\) can vary and the \(a_i\) may repeat.
We form finite products of elements of \(A\) because it would not make sense to form an infinite product of elements in a group.
These  finite products are called words.
Note that \(A\) is not required to be finite.
We convey \(\overline{\varnothing} = \{1\}\).
This way \(\overline{A}\) is never empty.

\begin{prop}
    If \(A\) is any subset of \(G\), then 
    \(\langle A\rangle = \overline{A}\).
\end{prop}

\begin{proof}
    We leave to the student to prove that \(\overline{A}\)  is a subgroup.
    It is clear that  \(A\subseteq \overline{A}\).
    Then  \(\langle A\rangle \subseteq \overline{A}\) since  \(\langle A\rangle\) is the smallest subgroup that contains \(A\) and \(\overline{A}\) is one of the  groups that contain \(A\).
    On the other hand, the product of any two elements of \(A\) belongs to \(\langle A\rangle\) because \(\langle A\rangle\) contains  \(A\) and it is closed under products.
    However, \(\overline{A}\) consists exactly of any finite product of elements of \(A\).
    Hence it easy follows \(\overline{A}\subseteq \langle A\rangle\).
    The proof is complete.
\end{proof}


\begin{rem}In light of this result, we write 
\[\langle A \rangle = \left\{ a_1^{k_1} a_2^{k_2} \cdots a_{n}^{k_n} \mid n\in \Z^+, a_i \in A ,\, k_i \in \Z \text{ and } a_i\neq a_{i+1} \text{ for any } 1 \leq i\leq  n \right\}\]
% \begin{enumerate}[label=(\roman*)]
% \item 

% % \item If \(G\) is Abelian, we can collect all the powers of \(a_i\), for each \(i\).
% % \item If \(A = \left\{ a_1, \ldots, a_n \right\}\subseteq G\) and  then \(\langle A\rangle  = \{a_1^{\alpha_1}, \ldots, a_n^{\alpha_n} \mid \alpha_i \in \Z\} \).
% \end{enumerate}
    
\end{rem}

\section{Normality, quotient groups and homomorphisms}

A useful reference for this section is Hungerford, chapter 1, section 5.

There are two standard groups associated to any group-homomorphism: its kernel and its image.
These are important concepts that you need to master.

\begin{defi}
    Let  \(\psi \colon G\to H\) be a morphism of groups.
    The kernel of \(\psi\) is 
    \[\ker \psi =\left\{ g\in G \mid \psi(g) = 1_H \right\}.\]
    The image of \(\psi\) is 
    \[\im\psi = \left\{ \psi(g) \mid g\in G \right\}\]
\end{defi}

\begin{eje}[Classwork]
    With \(\psi\) as above, prove \(\ker\psi \leq G\) and \(\im \psi \leq H \).
\end{eje}

\begin{prop}
    Let \(\psi\colon G\to H\) be a group-homomorphism.
    \begin{enumerate}[label=(\roman*)]
        \item \(\psi(1_G) = 1_H\)
        \item \(\psi(g^{-1}) = (\psi(g))^{-1}\)
        \item \(\psi(g^n) = (\psi(g))^n\)
    \end{enumerate}
\end{prop}

\begin{proof}
    See Dummit \& Foote, page 75.
\end{proof}

The only way to interpret \(\psi(g)^{-1}\) is as the inverse of \(\psi(g)\).
Thus we may drop the parenthesis in \((\psi(g))^{-1}\).


\begin{teo}\label{equivalent.conditions.normality}
Let \(N\leq G\).
The following conditions are equivalent.
\begin{enumerate}[label=(\roman*)]
\item Left congruence modulo \(N\) and right congruence modulo \(N\) define the same partition of \(G\).
\item For any \(g\in G\),\, \(Ng = gN\).
\item For any \(g\in G\),\, \(gNg^{-1}\subseteq N\). Here \(gNg^{-1} =\{  gxg^{-1} \mid x\in N\}\).
\item For any \(g\in G\),\, \(gNg^{-1} = N\). This means any \(g\in G\) normalizes \(N\).
\end{enumerate}
\end{teo}

\begin{defi}
    If \(N\leq G\) satisfies  \(gNg^{-1} = N\) for any \(g\in G\), then we say \(N\) is a normal subgroup of \(G\).
    In this case we use the notation \(N \unlhd G\).
\end{defi}

By the previous result, \(N\) is normal if it satisfies any of the equivalent conditions of Theorem \ref{equivalent.conditions.normality}.
The easiest way to verify a subgroup is normal is condition (iii).
Thus 
\[N\unlhd G\iff gNg^{-1}\subseteq N\] 
for any \(g\in G\).

\begin{prop}
    The kernel of any group-homomorphism is a normal subgroup.
\end{prop}

\begin{proof}
    Classwork.
\end{proof}

\begin{que}
    Is the image a normal subgroup?
\end{que}

\begin{teo}\label{thm:k.and.n.subgroups.of.G.and.N.normal}
    Let \(K\) and \(N\) be subgroups of a group \(G\) with \(N\unlhd G\). Then
    \begin{enumerate}[label=(\roman*)]
        \item \(N \cap K\unlhd K\) 
        \item \(N\unlhd N\vee K\)
        \item \(NK = N \vee K = KN\)
        \item If \(K\) is normal in \(G\) and \(K \cap N = \{e\}\), then \(nk = kn\) for all \(k \in K\) and \(n \in N\).
    \end{enumerate}
\end{teo}

\begin{eje}
    Provide examples that show when these conditions fail if \(N\) is not required to be normal in \(G\).
\end{eje}

\begin{proof}
\begin{enumerate}[label=(\roman*)]
    \item We have to prove that \(a (N\cap K) a^{-1}\subseteq N\cap K\) for any \(a\in K\).
     Let \(n \in N\cap K\) and \(a\in K\).
    Then \(ana^{-1}\in N\) because \(N\unlhd G\).
    Since \(n,a\in K\) and \(K\leq G\), we have \(ana^{-1}\in K\).
    Thus \(ana^{-1}\in N\cap K\).
    \item Trivial (Why? Note \(N \leq N \vee K\)\,)
    \item Exercise 
    \item Exercise
\end{enumerate}
\end{proof}


\begin{eje}
    Prove (iii) and (iv) of the preceding theorem.
\end{eje}
 
We have introduced normal subgroups for a reason: to make the quotient set of a group by a (normal) subgroup into a group.
In this way we can build new groups out of old.
Regarding the quotient set \(G/N\), two elements of  \(G\), say \(g\) and \(g'\) define the same equivalence class precisely when \(g' = gn\)  for some \(n\in N\), equivalently when \(g^{-1} g'  \in N\).
The condition that \(N\) be normal is precisely what we need to get a well-defined way of multiplying these equivalence classes.

\begin{teo}
    If \(N\unlhd G\), then 
    \[G/N = \left\{ \,xN \mid x \in G \,\right\}\]
    is a group under the operation \((xN)(yN) = (xy) N\).
    Moreover, the order of \(G/N\) is \(|G: N|\).
\end{teo}

\begin{proof}
It suffices to show that the operation is well-defined, that is, whenever we multiply two equivalence classes we must always get the same result no matter the representatives chosen.

If \(aN = xN\) and \(bN= yN\), then \(ax^{-1} = m \in N\) and \(by^{-1} = n \in N\) for some \(m,n\in N\).
Our goal is to prove that \(ab N = xy N\), i.e., that \((ab)(xy)^{-1}\in N\).
Note \[(ab)(xy)^{-1}=aby^{-1}x^{-1} = anx^{-1} = (an a^{-1})a x^{-1} =(an a^{-1}) m.\]
Since \(N\) is normal, \(aNa^{-1} \subseteq  N\) so \(ana^{-1} \in N\); and we already knew \(m\in N\).
Because \(N\) is closed under products, \((an a^{-1}) m \in N\).
This part of the  proof is complete.
The student should verify that the order of \(G/N\) is \(|G: N|\).
\end{proof}

You may want to take look at this \href{https://math.stackexchange.com/questions/14282/why-do-we-define-quotient-groups-for-normal-subgroups-only}{post}.


\begin{rem}
In additive notation, 
\begin{enumerate}[label=(\roman*)]
    \item \(G/N = \left\{ g+N \mid g\in G \right\}\)
    \item \((a+N)+(b+N) = (a+b) + N\)
\end{enumerate}
\end{rem}

The next result states that the kernel of any group-homomorphism is a normal subgroup, and that given normal subgroups occur as kernels.

\begin{teo}\hfill
\begin{enumerate}[label=(\roman*)]
    \item If \( f: G \to H \) is a group-homomorphism, then \( \ker f \trianglelefteq G \).
    \item Conversely, if \( N \trianglelefteq G \), then the map (called canonical projection) \( \pi: G \to G/N \) defined by \( a \mapsto aN \) is an surjective group-homomorphism with \[ \ker \pi = N .\]
\end{enumerate}
    

\end{teo}


\begin{proof}
\begin{enumerate}[label=(\roman*)]
    \item If \(x\in \ker f\) and \(a\in G\), then 
    \[f(axa^{-1}) = f(a)f(x)f(a^{-1}) = f(a)1_H f(a^{-1}) = 1_H\]
    meaning \(axa^{-1} \in \ker f\).
    Thus \(a \ker f a \subseteq \ker f\) for any \(a\in G\).

    \item Is is clear that \(\pi\) is surjective. (Make sure it is clear to you.) Further, \(\pi (ab) = ab N = (aN)(bN) = \pi (a) \pi (b)\) so \(\pi\) is a morphism of groups.
    Finally, 
    \begin{align*}
        \ker \pi &= \left\{ a\in G \mid \pi (a) = 1_{G/N} \right\}\\
        &= \left\{ a\in G \mid aN  = N \right\}\\
        &= N.
    \end{align*}
\end{enumerate}
The proof is complete.
\end{proof}

The next results tell us how to factor a group-homomorphism.

\begin{teo}\label{thm:dad.of.isomorphism.thms}
    If \( f: G \to H \) is a group homomorphism and \( N \trianglelefteq G \) is a subgroup contained in \( \ker f \), then there is a unique group-homomorphism \(\overline{f}\colon G/N \to H\) such that \(f = \overline{f}\circ \pi\), i.e., such that   the following diagram commutes.
    \[\begin{tikzcd}
        G \arrow[r, "f"] \arrow[d, "\pi"'] & H \\
        G/N \arrow[ru, "\overline{f}"']    &  
        \end{tikzcd}\]
    In addition, 
    \begin{enumerate}[label=(\roman*)]
        \item \(\im f = \im \overline{f}\),
        \item \(\ker \overline{f} = \ker f / N\), and 
        \item \(\overline{f} \) is an isomorphism if and only if \(f\) is an epimorphism and \(N = \ker f\).
    \end{enumerate}
\end{teo}

\begin{proof}
Define \(\overline{f}\colon G/N \to H: aN\mapsto f(a    )\).
Then \(\overline{f} \) is well-defined, for if \(aN = bN\), then \(ab^{-1}\in N\leq \ker f\), whence \(f(ab^{-1}) =  1_H\) and so \(f(a) = f(b)\).
Moreover \[\overline{f}((aN)(bN)) = \overline{f}(abN) = f(ab) = f(a)f(b) = \overline{f}(aN) \overline{f}(bN).\]
Finally, 
\begin{enumerate}[label=(\roman*)]
    \item  \(f(a)\in\im f \) if and only if  \(f(a) = \overline{f}(aN)  \in \im \overline{f} \). Hence \(\im f = \im \overline{f}\).
    % \item If \(aN \in \ker\overline{f}\), then \(\overline{f}(aN) = 1_H\), i.e, \(f(a) = 1_H\), so \(a\in \ker f\).
    % Thus \(aN \in \ker f / N\). The converse is clear.
\item 
    Note 
    \begin{align*}
        \ker \overline{f} &= \{x\in G/N \mid \overline{f}(x) = 1_H\}\\
        &= \left\{  aN \mid f(a) = 1_H \right\}\\
        &= \left\{ aN \mid a\in \ker f \right\}\\
        &= \ker f / N
    \end{align*}

    \item By (i), \(\overline{f} \) is epic if and only if \(f\) is.
     Note 
    \(\overline{f}\) is monic if and only if \(\ker \overline{f} = \left\{ 1_{G/N} \right\}   = \{ N\} \) if and only if \(\ker f / N  = \{N\}\) if and only if \(\ker f = N\).
    (Keep in mind that \(N\unlhd \ker f\) by hypothesis, and \(\ker f / N = N\)  implies \(aN  = N\) for all \(a \in \ker f\).) 
    Hence the result.
    % Moreover,
    % \(\overline{f} \) is an isomorphism if and only if \(\overline{f}\) is monic and epic, that is to say, 
\end{enumerate}
The proof is now complete.
\end{proof}


\begin{eje}
    Prove that if \(|G/N| = 1\), then \(G = N\).
\end{eje}

 
The important relationship between group-homomorphisms and quotient groups (also called factor groups) is a consequence of the result just proven.

\begin{cor}[First Isomorphism Theorem]
If \(f\colon G\to H\) is a group-homomorphism,  
\[G/ \ker f  \cong \im f.\]
    
\end{cor}

\begin{proof}
    Let \(\psi \colon G\to \im f : g\mapsto f(g) \).
    This way we force \(\psi\) to be surjective.
    Now apply  Theorem \ref{thm:dad.of.isomorphism.thms} with  \( \psi\) and \(N = \ker \psi\).
    Clearly \(\ker \psi = \ker f\).
    The following diagram commutes.
    \[\begin{tikzcd}
        G \arrow[r, "\psi"] \arrow[d, "\pi"'] & \im f \\
        G/\ker f   \arrow[ru, "\overline{f}"']    &  
        \end{tikzcd}\]
\end{proof}

The Diamond Theorem says that we can cancel out by paying off the intersection.

\begin{cor}[Second Isomorphism Theorem (Diamond Theorem)]
    If \(K,N \leq G\) and \(N\unlhd G\), then \[\frac{NK}{N}\cong \frac{K}{N\cap K}.\]
\end{cor}

\begin{proof}
We know \(N\unlhd NK\) by  Theorem \ref{thm:k.and.n.subgroups.of.G.and.N.normal}.
Let  \(f = \pi \circ \iota\), where 
\[\begin{tikzcd}
    K \arrow[r, "\iota", hook] & NK \arrow[r, "\pi", two heads] & \frac{NK}{N}
    \end{tikzcd}\]
Note \(f\) is a group-homomorphism with \(\ker f = K\cap N\).
Indeed,
\begin{align*}
\ker f &= \left\{ x\in K \mid f(x) = N \right\}\\
&= \left\{ x\in K \mid xN = N   \right\}\\
&= \left\{ x\in K\mid x\in N \right\}\\
&= K\cap N,
\end{align*}
where we have used \[xN = N \;\text{ if and only if }\; x\in N.\]
Let us see \(f\) is epic.
Let \(nkN \in NK/N\).
The normality of \(N\) implies that \(nkN = Nnk = Nk\), so  \[nkN  =Nk = f(k).\]
Hence \(\im f = NK/ N\) and  by the First Isomorphism Theorem, 
\[\frac{K}{N\cap K}\cong \frac{NK}{N}.\]
End of the proof.
\end{proof}


\begin{cor}[Third Isomorphism Theorem]
    If \(H,K\unlhd G\) and \(K\leq H\), then 
    \[\frac{H}{K}  \unlhd \frac{G}{K} \quad\text{and}\quad \frac{G/ K}{H/K}\cong \frac{G}{H}.\]
\end{cor}

\begin{proof}
\begin{enumerate}[label=(\roman*)]
    \item Prove that \(\frac{H}{K}  \unlhd \frac{G}{K}\).
    \item Define \(\psi\colon G/K \to G/H: gK \mapsto gH\) and prove \(\psi\) is a well-defined epic homomorphism.
    \item By the First Isomorphism Theorem, \[\frac{G/K}{\ker \psi}\cong G/H.\]
    \item Notice that 
    \begin{align*}
        \ker \psi &= \left\{ gK \mid g\in G,\, \psi (gK) = 1_{G/H} \right\}\\
        &= \left\{ gK \mid g\in G,\, g H = H \right\}\\
        &= \left\{ gK \mid g\in H \right\}\\
        &= H / K    
    \end{align*}
    and conclude.
\end{enumerate}
\end{proof}

There is still one more isomorphism theorem.

\begin{cor}
Let  \(N\unlhd G\).
There is a one-to-one correspondence between subgroups of \(G\) that contain \(N\) and subgroups of \(G/N\).
In particular, every subgroup of \(G/N\) is of the form \(A/N\) with \(N\leq A\leq G\).
Furthermore, for all \(A,B\leq G\) such that \(N\leq A\) and \(N\leq B\), it holds
\begin{enumerate}[label=(\roman*)]
    \item \(A\leq B\) if and only if \(A/N \leq B/N\), and 
    \item \(A\unlhd G\) if and only if \(A/N\unlhd G/N\).
\end{enumerate}
\end{cor}

\begin{proof}
    Trivialito. (If it is not clear, then the proof is left as  an exercise.)
\end{proof}

\begin{eje}
With the notations as in the statement of the Fourth Isomorphism Theorem, prove 
\begin{enumerate}[label=(\roman*)]
    \item if \(A\leq B\), then \(|B:A| = |B/N : A/ N|\)
    \item \({\langle A,B  \rangle} / N = \langle A/N, B/N \rangle \)
    \item \(({A\cap B})/N = A / N \cap B/ N\)
\end{enumerate}
\end{eje}

Here ends  the group theory that you will see in  this course.
\end{document}