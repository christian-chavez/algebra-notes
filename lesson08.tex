\documentclass[11pt,a4paper]{article}

% PAQUETES
\usepackage[T1]{fontenc}%
\usepackage[utf8]{inputenc}%

\usepackage[english]{babel}
\usepackage{amsmath}
\usepackage{amsthm}
\usepackage{amsfonts}
\usepackage[%left=1.54cm,right=1.54cm,top=1.54cm,bottom=1.54cm
    margin=1in, includefoot,
]{geometry}
\usepackage{xfrac}  
\usepackage{tikz-cd}
\usepackage{enumerate}
\usepackage{amsfonts}
\usepackage{amssymb}
\usepackage{tcolorbox}
\usepackage{rotating}
\usepackage{mathpazo}
% \usepackage{charter}
\usetikzlibrary{babel}
\usepackage{listings}
\usepackage{amssymb}
\usepackage{extarrows}
\usepackage{makeidx}
\usepackage{graphicx}
\usepackage{multirow}
\usepackage{tikz-cd}
\usepackage{tasks}
\usepackage{xcolor}
\usepackage{mathrsfs} % 2024-11-04

% Christian
\usepackage{enumitem,etoolbox,titlesec}


%OPERADORES

\DeclareMathOperator{\dom}{dom}
\DeclareMathOperator{\cod}{cod}
\DeclareMathOperator{\id}{id}

\newcommand{\red}[1]{\textcolor{red}{#1}}
\renewcommand{\ker}{\operatorname{Ker}}
\newcommand{\im}{\operatorname{Im}}
\newcommand{\C}{\mathbb{C}}
\newcommand{\R}{\mathbb{R}}
\newcommand{\Q}{\mathbb{Q}}
\newcommand{\N}{\mathbb{N}}
\newcommand{\Z}{\mathbb{Z}}
\newcommand{\D}{\mathbb{D}}
\newcommand{\B}{\mbox{Ob}}
\newcommand{\M}{\mbox{Mo}}
\newcommand{\del}{\Delta}
\newcommand{\odel}[1]{\left[#1\right]}
\newcommand{\Hom}[1]{\text{Hom}(#1)}
\newcommand{\adel}[1]{\left\lbrace #1 \right\rbrace}
%\renewcommand{\theequation}{\thesection.\arabic{equation}}
\newcommand{\funcion}[5]{%
{\setlength{\arraycolsep}{2pt}
\begin{array}{r@{}ccl}
#1\colon \hspace{0pt}& #2 & \longrightarrow & #3\\
& #4 & \longmapsto & #5
\end{array}}}

\newcommand{\func}[3]{#1\colon  #2  \to  #3}

\newcommand\restr[2]{{% we make the whole thing an ordinary symbol
  \left.\kern-\nulldelimiterspace % automatically resize the bar with \right
  #1 % the function
  \vphantom{\big|} % pretend it's a little taller at normal size
  \right|_{#2} % this is the delimiter
  }}

%ENTORNOS

% \theoremstyle{theorem}
\newtheorem{teo}{Theorem}[section]
\newtheorem{prop}[teo]{Proposition}
\newtheorem{lem}[teo]{Lemma}
\newtheorem{cor}[teo]{Corollary}

\theoremstyle{definition}
\newtheorem{defi}[teo]{Definition}
\newtheorem{rem}{Remark}[teo]
\newtheorem{exa}{Example}
\newtheorem{eje}{Exercise}
\newtheorem{que}{Question}

\newenvironment{sol}
  {\begin{proof}[\textit{Solution}]}
  {\end{proof}}

\titlelabel{\thetitle.\quad}

\def\contador{}
\graphicspath{{./figures/}}
\newcommand{\qand}{\quad\text{and}\quad}
\usepackage{microtype,parskip}
\setlength{\parindent}{15pt}
\linespread{1.15}
\usepackage{hyperref}
\hypersetup{
    colorlinks=true,  
    allcolors=blue,
    pdfproducer={Christian Chávez},
}

\makeatletter
\@ifclassloaded{exam}{
    \footer{}{\thepage}{}
    \renewcommand{\thequestion}{\bfseries\arabic{question}}
    \renewcommand{\solutiontitle}{\noindent\textit{Solution.}\enspace}
    \unframedsolutions
}{}
\makeatother


\newenvironment{theproof}
{
    \renewcommand{\solutiontitle}{}
    \begin{solution}
    \vspace*{-\baselineskip}
    \begin{proof}
}
{
    \end{proof}
    \end{solution}
    \renewcommand{\solutiontitle}{\noindent\textit{Solution.} }
}

\usepackage[style=numeric]{biblatex}
\addbibresource{bibliography.bib}

\begin{document}

\def\contador{Lesson 8}
\noindent
\begin{minipage}[c]{0.33\textwidth}
    \includegraphics[scale=0.37]{sello_yachay.png}
\end{minipage}
\begin{minipage}[c]{0.37\textwidth}
    % \centering
    \textbf{\large School of Mathematical and\\ Computational Sciences}\par
    Abstract Algebra
\end{minipage}
~ 
\begin{minipage}[c]{3mm}
    \raggedleft
    \rule[1.5mm]{0.3mm}{15mm}
\end{minipage}
~ 
\begin{minipage}[c]{0.24\textwidth}
    \raggedleft
    Prof. Pablo Rosero\\
    \& Christian Chávez\\
    \contador{}
\end{minipage}

\vspace{1mm}
\noindent\hrulefill

\vspace{3mm}


\section{Subgroup generated by subsets of a group}

\begin{prop}
    If \(\mathcal{A}\) is any nonempty collection of subgroups of \(G\), then 
    \[\bigcap_{H\in \mathcal{A}} H \leq G.\]
\end{prop}



\begin{defi}
Let \(A\) be any subset of \(G\).
The \textbf{subgroup generated by} \(A\) is 
\[\langle A\rangle := \bigcap_{\stackrel{A\subseteq H }{H\leq G }  } H .\]
\end{defi}

This definition says that \(\langle A\rangle \) is the smallest subgroup of \(G\) that contains \(A\). 
It is clear that if the subgroup generated by a subgroup \(H\) is \(H\) itself. 
What would be the subgroup generated by \(\varnothing\)?


\begin{rem}
\begin{enumerate}[label=(\roman*)]
    \item      If \(A \) is a finite set, say \(A = \{a_1, \ldots, a_n \}\),
    then we simply write \(\langle A\rangle = \langle a_1, \ldots, a_n \rangle \)

    \item Recall from the previous lesson that \(\langle a\rangle\) denotes the cyclic subgroup generated by \(a\).
    With the definition above, it is easy to see that this is the same as the subgroup generated by \(\{a\}\).
    Thus the notation is unambiguous.

    \item If \(A,B\subset G\), then we write  \(\langle A, B \rangle\) to mean  \(\langle A\cup B\rangle \).
    This subgroup is denoted \(A \vee B\). 
\end{enumerate}
\end{rem}
 
\begin{defi}
    Let \(A\subset G\).
    Define 
    \[\overline{A} = \left\{ a_1^{k_1} a_2^{k_2} \cdots a_{n}^{k_n} \mid n\in \Z_0^+, a_i \in A ,\, k_i = \pm 1 \text{ for all } 0 \leq i\leq n \right\}\]
\end{defi}

Note that \(n\) can vary and the \(a_i\) may repeat.
We form finite products of elements of \(A\) because it would not make sense to form an infinite product of elements in a group.
These  finite products are called words.
Note that \(A\) is not required to be finite.
We convey \(\overline{\varnothing} = \{1\}\).
This way \(\overline{A}\) is never empty.

\begin{prop}
    If \(A\) is any subset of \(G\), then 
    \(\langle A\rangle = \overline{A}\).
\end{prop}

\begin{proof}
    We leave to the student to prove that \(\overline{A}\)  is a subgroup.
    Now it is clear that  \(A\subseteq \overline{A}\).
    Then  \(\langle A\rangle \subseteq \overline{A}\) since  \(\langle A\rangle\) is the smallest subgroup that contains \(A\) and \(\overline{A}\) is one of the  groups that contain \(A\).
    On the other hand, the product of any two elements of \(A\) belongs to \(\langle A\rangle\) because \(\langle A\rangle\) contains  \(A\) and it is closed under products.
    However, \(\overline{A}\) consists exactly of any finite product of elements of \(A\).
    Hence it easy follows \(\overline{A}\subseteq \langle A\rangle\).
    The proof is complete.
\end{proof}


\begin{rem}
\begin{enumerate}[label=(\roman*)]
\item In light of this result, we write 
\[\langle A \rangle = \left\{ a_1^{k_1} a_2^{k_2} \cdots a_{n}^{k_n} \mid n\in \Z^+, a_i \in A ,\, k_i \in \Z \text{ and } a_i\neq a_{i+1} \text{ for any } 1 \leq i\leq  n \right\}\]

% \item If \(G\) is Abelian, we can collect all the powers of \(a_i\), for each \(i\).
% \item If \(A = \left\{ a_1, \ldots, a_n \right\}\subseteq G\) and  then \(\langle A\rangle  = \{a_1^{\alpha_1}, \ldots, a_n^{\alpha_n} \mid \alpha_i \in \Z\} \).
\end{enumerate}
    
\end{rem}

\section{Normality, quotient groups and homomorphisms}

A useful reference for this section is Hungerford, chapter 1, section 5.

There are two standard groups associated to any group-homomorphism: its kernel and its image.
\begin{defi}
    Let  \(\psi \colon G\to H\) be a morphism of groups.
    The kernel of \(\psi\) is 
    \[\ker \psi =\left\{ g\in G \mid \psi(g) = 1_H \right\}.\]
    The image of \(\psi\) is 
    \[\im\psi = \left\{ \psi(g) \mid g\in G \right\}\]
\end{defi}

\begin{eje}[Classwork]
    With \(\psi\) as above, prove \(\ker\psi \leq G\) and \(\im \psi \leq H \).
\end{eje}

\begin{prop}
    Let \(\psi\colon G\to H\) be a group-homomorphism.
    \begin{enumerate}[label=(\roman*)]
        \item \(\psi(1_G) = 1_H\)
        \item \(\psi(g^{-1}) = (\psi(g))^{-1}\)
        \item \(\psi(g^n) = (\psi(g))^n\)
    \end{enumerate}
\end{prop}

\begin{proof}
    See Dummit \& Foote, page 75.
\end{proof}

The only way to interpret \(\psi(g)^{-1}\) is as the inverse of \(\psi(g)\).
Thus we may drop the parenthesis in \((\psi(g))^{-1}\).


\begin{teo}\label{equivalent.conditions.normality}
Let \(N\leq G\).
The following conditions are equivalent.
\begin{enumerate}[label=(\roman*)]
\item Left congruence modulo \(N\) and right congruence modulo \(N\) define the same partition of \(G\).
\item For any \(g\in G\),\, \(Ng = gN\).
\item For any \(g\in G\),\, \(gNg^{-1}\subseteq N\). Here \(gNg^{-1} =\{  gxg^{-1} \mid x\in N\}\).
\item For any \(g\in G\),\, \(gNg^{-1} = N\). This means any \(g\) normalizes \(N\).
\end{enumerate}
\end{teo}

\begin{defi}
    If \(N\leq G\) satisfies  \(gNg^{-1} = N\) for any \(g\in G\), then we say \(N\) is a normal subgroup of \(G\).
    In this case we use the notation \(N \unlhd G\).
\end{defi}

By the previous result, \(N\) is normal if it satisfies any of the equivalent conditions of Theorem \ref{equivalent.conditions.normality}.
The easiest way to verify a subgroup is normal is condition (iii).
Thus 
\[N\unlhd G\iff gNg^{-1}\subseteq N\] 
for any \(g\in G\).

\begin{teo}
    Let \(K\) and \(N\) be subgroups of a group \(G\) with \(N\unlhd G\). Then
    \begin{enumerate}[label=(\roman*)]
        \item \(N \cap K\unlhd K\) 
        \item \(N\unlhd N\vee K\)
        \item \(NK = N \vee K = KN\)
        \item If \(K\) is normal in \(G\) and \(K \cap N = \{e\}\), then \(nk = kn\) for all \(k \in K\) and \(n \in N\).
    \end{enumerate}
\end{teo}

\begin{eje}
    Provide examples that show when these conditions fail if \(N\) is not required to be normal in \(G\).
\end{eje}

\begin{proof}
\begin{enumerate}[label=(\roman*)]
    \item We have to prove that \(a (N\cap K) a^{-1}\subseteq N\cap K\) for any \(a\in K\).
     Let \(n \in N\cap K\) and \(a\in K\).
    Then \(ana^{-1}\in N\) because \(N\unlhd G\).
    Since \(n,a\in K\) and \(K\leq G\), we have \(ana^{-1}\in K\).
    Thus \(ana^{-1}\in N\cap K\).
    \item Trivial (Why? Note \(N \leq N \vee K\)\,)
    \item Exercise 
    \item Exercise
\end{enumerate}

\begin{eje}
    Prove (iii) and (iv) of the preceeding theorem.
\end{eje}

\end{proof}
 
We have introduced normal groups for a reason: to make the quotient set of a group by a (normal) subgroup into a group.
In this way we can build new groups out of old.

\begin{teo}
    If \(N\unlhd G\), then 
    \[G/N = \left\{ \,xN \mid x \in G \,\right\}\]
    is a group under the operation \((xN)(yN) = (xy) N\).
    Moreover, the order of \(G/N\) is \(|G: N|\).
\end{teo}

\begin{proof}
It suffices to show that the operation is well-defined, that is, 

\end{proof}

\begin{teo}
    
\end{teo}


\begin{proof}
    
\end{proof}


\begin{rem}
    
\end{rem}

\begin{teo}
    
\end{teo}

\begin{proof}
    
\end{proof}


\begin{teo}
    
\end{teo}


\begin{proof}
    
\end{proof}

\begin{cor}
    
\end{cor}

\begin{proof}
    
\end{proof}

\begin{cor}
    
\end{cor}

\begin{proof}
    
\end{proof}

\end{document}