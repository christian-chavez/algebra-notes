\documentclass[11pt,a4paper]{article}

% PAQUETES
\usepackage[T1]{fontenc}%
\usepackage[utf8]{inputenc}%

\usepackage[english]{babel}
\usepackage{amsmath}
\usepackage{amsthm}
\usepackage{amsfonts}
\usepackage[%left=1.54cm,right=1.54cm,top=1.54cm,bottom=1.54cm
    margin=1in, includefoot,
]{geometry}
\usepackage{xfrac}  
\usepackage{tikz-cd}
\usepackage{enumerate}
\usepackage{amsfonts}
\usepackage{amssymb}
\usepackage{tcolorbox}
\usepackage{rotating}
\usepackage{mathpazo}
% \usepackage{charter}
\usetikzlibrary{babel}
\usepackage{listings}
\usepackage{amssymb}
\usepackage{extarrows}
\usepackage{makeidx}
\usepackage{graphicx}
\usepackage{multirow}
\usepackage{tikz-cd}
\usepackage{tasks}
\usepackage{xcolor}
\usepackage{mathrsfs} % 2024-11-04

% Christian
\usepackage{enumitem,etoolbox,titlesec}


%OPERADORES

\DeclareMathOperator{\dom}{dom}
\DeclareMathOperator{\cod}{cod}
\DeclareMathOperator{\id}{id}

\newcommand{\red}[1]{\textcolor{red}{#1}}
\renewcommand{\ker}{\operatorname{Ker}}
\newcommand{\im}{\operatorname{Im}}
\newcommand{\C}{\mathbb{C}}
\newcommand{\R}{\mathbb{R}}
\newcommand{\Q}{\mathbb{Q}}
\newcommand{\N}{\mathbb{N}}
\newcommand{\Z}{\mathbb{Z}}
\newcommand{\D}{\mathbb{D}}
\newcommand{\B}{\mbox{Ob}}
\newcommand{\M}{\mbox{Mo}}
\newcommand{\del}{\Delta}
\newcommand{\odel}[1]{\left[#1\right]}
\newcommand{\Hom}[1]{\text{Hom}(#1)}
\newcommand{\adel}[1]{\left\lbrace #1 \right\rbrace}
%\renewcommand{\theequation}{\thesection.\arabic{equation}}
\newcommand{\funcion}[5]{%
{\setlength{\arraycolsep}{2pt}
\begin{array}{r@{}ccl}
#1\colon \hspace{0pt}& #2 & \longrightarrow & #3\\
& #4 & \longmapsto & #5
\end{array}}}

\newcommand{\func}[3]{#1\colon  #2  \to  #3}

\newcommand\restr[2]{{% we make the whole thing an ordinary symbol
  \left.\kern-\nulldelimiterspace % automatically resize the bar with \right
  #1 % the function
  \vphantom{\big|} % pretend it's a little taller at normal size
  \right|_{#2} % this is the delimiter
  }}

%ENTORNOS

% \theoremstyle{theorem}
\newtheorem{teo}{Theorem}[section]
\newtheorem{prop}[teo]{Proposition}
\newtheorem{lem}[teo]{Lemma}
\newtheorem{cor}[teo]{Corollary}

\theoremstyle{definition}
\newtheorem{defi}[teo]{Definition}
\newtheorem{rem}{Remark}[teo]
\newtheorem{exa}{Example}
\newtheorem{eje}{Exercise}
\newtheorem{que}{Question}

\newenvironment{sol}
  {\begin{proof}[\textit{Solution}]}
  {\end{proof}}

\titlelabel{\thetitle.\quad}

\def\contador{}
\graphicspath{{./figures/}}
\newcommand{\qand}{\quad\text{and}\quad}
\usepackage{microtype,parskip}
\setlength{\parindent}{15pt}
\linespread{1.15}
\usepackage{hyperref}
\hypersetup{
    colorlinks=true,  
    allcolors=blue,
    pdfproducer={Christian Chávez},
}

\makeatletter
\@ifclassloaded{exam}{
    \footer{}{\thepage}{}
    \renewcommand{\thequestion}{\bfseries\arabic{question}}
    \renewcommand{\solutiontitle}{\noindent\textit{Solution.}\enspace}
    \unframedsolutions
}{}
\makeatother


\newenvironment{theproof}
{
    \renewcommand{\solutiontitle}{}
    \begin{solution}
    \vspace*{-\baselineskip}
    \begin{proof}
}
{
    \end{proof}
    \end{solution}
    \renewcommand{\solutiontitle}{\noindent\textit{Solution.} }
}

\usepackage[style=numeric]{biblatex}
\addbibresource{bibliography.bib}

\begin{document}

\def\contador{Lesson 10}
\noindent
\begin{minipage}[c]{0.33\textwidth}
    \includegraphics[scale=0.37]{sello_yachay.png}
\end{minipage}
\begin{minipage}[c]{0.37\textwidth}
    % \centering
    \textbf{\large School of Mathematical and\\ Computational Sciences}\par
    Abstract Algebra
\end{minipage}
~ 
\begin{minipage}[c]{3mm}
    \raggedleft
    \rule[1.5mm]{0.3mm}{15mm}
\end{minipage}
~ 
\begin{minipage}[c]{0.24\textwidth}
    \raggedleft
    Prof. Pablo Rosero\\
    \& Christian Chávez\\
    \contador{}
\end{minipage}

\vspace{1mm}
\noindent\hrulefill

\vspace{3mm}

\section{Ideals,  \& Quotient Rings}
Homomorphisms
\subsection{Ideals}

Let \((A,+,\cdot)\) be a ring (not necessarily with unity).
Recall that \(S\) a subring of \(A\)  if   \((S,+)\) is a subgroup of \((A,+)\) and \(S\cdot S \subseteq S\), i.e., \(S\) is closed under multiplication. 
In other words, \(S\) is  a   subring of \(A\) if \(S\) is a subset of \(A\) that  together with the operations of \(A\) is itself a ring.
Notice that, if \(A\) is a ring with unity,  we do not require subrings of \(A\) to  contain the unity \(1_A\).

The beginning student may think that a subring plays the same importance of the analogue concept of subgroup in group theory.
This is not the case.
To define the important notion of quotient ring, for example, subrings are not suitable.
Instead, we need a more apropiate notion: that of \textit{ideal}.
We will focus on studying ideals and their properties.

\begin{defi}
    An \textbf{ideal} of a ring \(A\) is a subset \(I\subseteq A\) such that 
    \begin{enumerate}[label=(\roman*)]
        \item \((I,+)\leq (A,+)\),  
        \item \(a I \subseteq I\) and \(Ia \subseteq I\) for every \(a\in A\).
    \end{enumerate}
\end{defi}

The condition \(aI\subseteq I\)  means that if \(a\in A\) and \(b \in I\), then \(ab \in I\). Recall that \(aI = \left\{ a b \mid b\in I \right\}\)
and  \(I a = \left\{  ba  \mid b\in I \right\}\).
This automatically implies that \(I\) is closed under multiplication, as the reader should check.
Thus, any   ideal is a subring.

If \(J\subseteq A\) verifies 
\[a J \subseteq J\quad\text{for every }a\in A\]
we say \(J\) absorbs products from the left.
We say \(J\) absorbs products from the right with the obvious modification.
% If \(J\) absorbs products from left and from the right we just say \(J\) absorbs products.
Therefore, an ideal is an addive subgroup that absorbs products from left and from the right.

\begin{rem}
    A \textbf{left ideal} of a ring \(A\) is an addive subgroup of \((A,+)\) that absorbs products from the left.
    Similarly,
    a \textbf{right ideal}  is an addive subgroup of \((A,+)\) that absorbs products from the right.
    Thus, a \textbf{two sided ideal}, or ideal for short,
    is an additive subgroup that is both a left and right ideal.
    If \(A\) is commutative, all these notions are the same.
\end{rem}

\begin{exa}
\begin{enumerate}[label=(\roman*)]
\item For any ring \(A\), both \(\left\{ 0 \right\} \) and \(A\) are ideals of \(A\). We call \(\left\{ 0 \right\}\) the trivial ideal of \(A\), and it is usually denoted \(0\).
\item \(n\Z\) is an ideal of \(\Z\) for every \(n\in \Z\).
\item Given integers \(0\leq n < m\), the set \(n \Z_m\) is an ideal of \(\Z_m\).
\item The set  $$I=\left\{\left[\begin{array}{ll}a & 0 \\ b & 0\end{array}\right]: a, c \in A\right\}$$
is a left-ideal but not a right-ideal of the ring of \(2\times 2\) matrices with entries in a ring \(A\).
\item Let $R$ be a commutative ring with unity and let $a \in R$. The set $$\langle a\rangle=\{r a \mid r \in R\}$$ is an ideal of $R$ called the principal ideal generated by $a$.  
\item  Let $\mathbb{R}[x]$ denote the set of all polynomials with real coefficients and let $A$ denote the subset of all polynomials with constant term 0 . Then $A$ is an ideal of $\mathbb{R}[x]$. In fact, $A=\langle x\rangle$.
\item  Let $R$ be the ring of all real-valued functions of a real variable. The subset  of all differentiable functions is a subring of $R$ but not an ideal of $R$.
\item Let $T$ be the ring of all functions from $\mathbb{R}$ to $\mathbb{R}$. Let $I$ be the subset consisting of those functions $g$ such that $g(2)=0$. Then $I$ is a subring of $T$. Furthermore, if $f\in T$   and   $g \in I$, then
\[
(f g)(2)=f(2) g(2)=f(2) \cdot 0=0
\]
Therefore, $f g \in I$. Similarly, $g f \in I$, so that $I$ is an ideal in $T$.
\end{enumerate}
\end{exa}

\begin{eje}
Show the following are equivalent for a subset \(J\subseteq \Z\):
\begin{enumerate}[label=(\roman*)]
\item \(J\) is a subgroup of \(\Z\)
\item \(J\) is an ideal of \(\Z\)
\item \(J = n \Z\) for some \(n\in \Z\)
\end{enumerate}
\end{eje}

\end{document}