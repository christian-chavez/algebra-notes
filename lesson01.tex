\documentclass[11pt,a4paper]{article}

% PAQUETES

\usepackage[spanish]{babel}
\usepackage{amsmath}
\usepackage{amsthm}
\usepackage{amsfonts}
\usepackage[left=1.54cm,right=1.54cm,top=1.54cm,bottom=1.54cm]{geometry}
\usepackage{xfrac}  
\usepackage{tikz-cd}
\usepackage{enumerate}
\usepackage{amsfonts}
\usepackage{amssymb}
\usepackage{tcolorbox}
\usepackage{rotating}
\usepackage{mathpazo}
\usepackage{charter}
\usetikzlibrary{babel}
\usepackage{listings}
\usepackage{amssymb}
\usepackage{extarrows}
\usepackage{makeidx}
\usepackage{graphicx}
\usepackage{multirow}
\usepackage{tikz-cd}
\usepackage{tasks}
\usepackage{color}


%OPERADORES

\DeclareMathOperator{\dom}{dom}
\DeclareMathOperator{\cod}{cod}
\DeclareMathOperator{\id}{id}


\newcommand{\C}{\mathbb{C}}
\newcommand{\im}{\text{Im}}
\newcommand{\R}{\mathbb{R}}
\newcommand{\N}{\mathbb{N}}
\newcommand{\Z}{\mathbb{Z}}
\newcommand{\D}{\mathbb{D}}
\newcommand{\B}{\mbox{Ob}}
\newcommand{\M}{\mbox{Mo}}
\newcommand{\del}{\Delta}
\newcommand{\odel}[1]{\left[#1\right]}
\newcommand{\Hom}[1]{\text{Hom}(#1)}
\newcommand{\adel}[1]{\left\lbrace #1 \right\rbrace}
%\renewcommand{\theequation}{\thesection.\arabic{equation}}
\newcommand{\funcion}[5]{%
{\setlength{\arraycolsep}{2pt}
\begin{array}{r@{}ccl}
#1\colon \hspace{0pt}& #2 & \longrightarrow & #3\\
& #4 & \longmapsto & #5
\end{array}}}

\newcommand{\func}[3]{#1\colon  #2  \to  #3}

\newcommand\restr[2]{{% we make the whole thing an ordinary symbol
  \left.\kern-\nulldelimiterspace % automatically resize the bar with \right
  #1 % the function
  \vphantom{\big|} % pretend it's a little taller at normal size
  \right|_{#2} % this is the delimiter
  }}

%ENTORNOS

\theoremstyle{theorem}
\newtheorem{teo}{Theorem}[section]
\newtheorem{prop}{Proposition}[section]
\newtheorem{lem}{Lemma}[section]
\newtheorem{cor}{Corollary}[section]

\theoremstyle{definition}
\newtheorem{defi}{Definition}[section]
\newtheorem{rem}{Remark}
\newtheorem{eje}{Exercise}
\newtheorem{sol}{\textit{Solution}}
\newtheorem{que}{Question}


\def\contador{}
\graphicspath{{./figures/}}


\begin{document}

\def\contador{1}
\begin{minipage}[c]{30mm}
\includegraphics[scale=0.5]{sello_yachay.png}
\end{minipage}
\begin{minipage}[c]{95mm}
\textbf{School of Mathematical and Computational Sciences}
\end{minipage}
\begin{minipage}[c]{2mm}
\rule[2.5mm]{0.3mm}{15mm}
\end{minipage}
\begin{minipage}[c]{50mm}
Prof. Pablo Rosero.\\ Abstract Algebra: Lesson \contador{}
\end{minipage}

\hrulefill
\vspace{3mm}

\section{Basic properties of the integers}

In this lesson and onwards, we consider $\Z$ to be the set of integers numbers, whereas $\Z^{+}$ is the set of strictly positive integers numbers.

\begin{defi}
Let $a,b\in \Z$, with $a\neq 0$. We say  $a$ is a divisor of $b$ if there is an integer $c$ such that $a\cdot c=b$. In this case, we write $a\mid b.$ 
\end{defi}

\begin{rem}
    If $a$ does not divide $b$, we write $a\nmid b$.
\end{rem}

\begin{teo}
    Let $a,b\in \Z\setminus\{0\}$, there is a unique positive integer $d$, called the \textbf{greatest common divisor of $a$ and $b$}, satisfying 
    \begin{enumerate}
        \item $d\mid a$ and $d\mid b.$
        \item If $e\mid a$ and $e\mid b$ then $e\mid d.$
    \end{enumerate}
\end{teo}

\begin{rem}
    If $d$ is the greatest common divisor of $a$ and $b$, we write $d=(a,b).$
    In the  particular case when  $(a,b)=1$, we say  $a$ and $b$ are coprimes.
\end{rem}

\begin{que}
    Why does $(a,b)$ always exist for $a,b\in \Z\setminus\{0\}$?
\end{que}


\begin{teo}[Division algorithm]
    If \(a,b\in \Z\setminus\{0\}\), 
    there are unique \(q,r\in Z\)
    such that 
    \[a= qb +r \qand 0\leq r < |b|.\]
    We call \(q\) the quotient and \(r\) the remainder.
\end{teo}

\begin{proof}
    
\end{proof}



\paragraph{Euclidean Algortihm.} This is an efficient method to compute the gcd of any  two integers.
It is based on the division algorithm.
(Keep in mind that, despite the name, the \textit{division algorithm} is a theorem whereas the \textit{euclidean algorithm} is a procedure.)

If \(a\) and \(b\) are nonzero integers, then by the division algorithm we get \(q,r\in \Z\) such that \(a= qb+r\).
Let \(q_0=q\) and \(r_0=r\).
By applying the division algorithm again with \(q_0\) and \(r_0\) we obtain a new quotient \(q_1\)   and a new remainder \(r_1\).
The idea of this procedure is to continue applying the division algorithm until we reach a zero remainder. From one step to the next,  the divisor becomes the dividend and the remainder the divisor, as follows:
\begin{equation}\label{eqtn:sequence.euclidean.algorithm}
\begin{aligned}
    a & =q_0 b+r_0 \\
    b & =q_1 r_0+r_1 \\
    r_0 & =q_2 r_1+r_2 \\
    r_1 & =q_3 r_2+r_3 \\
    &\;\; \vdots \\
    r_{n-2} & =q_n r_{n-1}+r_n \\
    r_{n-1} & =q_{n+1} r_n
\end{aligned}
\end{equation}


\begin{que}
    Why the Euclidean algorithm always terminates? In other words,
    why we always get a zero remainder at the end of the Euclidean algorithm? 
    Keep in mind the condition \(0\leq r < |b|\) in the division algorithm.
\end{que}

As a consequence of the Euclidean algorithm,
the greatest common divisor of two integers can be written as a linear combination of those integers. This can be done by backward substitution in \eqref{eqtn:sequence.euclidean.algorithm}.

\begin{teo}[Bézout's identity]
    Let $a$ and $b$ be integers with $d=(a,b)$. Then there exist integers $x$ and $y$ such that $a x+b y=d$. 
\end{teo}

\begin{eje}
    Compute \((1761, 1567)\) and write this integer as a linear combination of \(1761\) and \(1567\).
\end{eje}

\begin{sol} By the Euclidean algorithm,
   \begin{align*}
    1761 &= 1\cdot 1567 + 194\\
    1567 &= 8\cdot194 + 15\\
    194  &= 12\cdot 15 + 14\\
    15   &= 1\cdot 14 + 1 \\
    14   &= 14\cdot 1 + 0.
   \end{align*}
   From the next to last line we get \((1761,1567) = 1\).
\end{sol}

\begin{defi}
    An integer \(p\) is prime iff 
    \begin{enumerate}[label=(\roman*)]
        \item \(p>1\), and
        \item the only positive divisors of \(p\) are \(p\) and \(1\).
    \end{enumerate}
    A \textit{composite} integer is an integer greater than \(1\) that is not prime.
\end{defi}

Thus, every positive integer is composite, prime, or the unit 1.

\begin{rem}\label{rmk:gcd.of.prime.and.integer}
    If $p$ is a prime and $b \in \mathbb{Z}\setminus \{0\}$ then
\[
(p, b)= \begin{cases}p & \text { if } p \mid b, \\ 1 & \text { else.}\end{cases}
\]
Prove this claim.
\end{rem}

\begin{prop}
    Let \(I\subseteq \Z\) be such that 
    \begin{enumerate}[label=(\roman*)]
        \item\label{lab:i.exercise.ideal.of.Z} \(0\in I\),
        \item\label{lab:ii.exercise.ideal.of.Z} if  \(a,b\in I\), then \(a-b\in I\), 
        \item\label{lab:iii.exercise.ideal.of.Z} if \(a\in I\) and \(q\in I\), then  \(aq\subseteq I\).
    \end{enumerate}
    Then, there is some nonnegative integer \(d\in I\) such that  \[I = \left\{ dk: k \in \Z \right\}.\]
\end{prop}

\begin{rem}
    If \(A\subseteq \Z\) and \(n\in \Z\), we denote 
    \(nA = \left\{n a : a\in A \right\}\).
    If \(A=\Z\), then \((n) = n\Z\).
    Thus, this result states that \(I=(d)\) for some \(d\in I\).
\end{rem}

\begin{proof}
    Condition \ref{lab:i.exercise.ideal.of.Z} states \(I\neq \varnothing\).
   If \(I = \left\{ 0 \right\}\), take \(d=0\).
   Suppose \(I\neq \left\{ 0 \right\}\)  and \(a\in I\). 
   By \ref{lab:ii.exercise.ideal.of.Z}, 
   if \(a\in I\), then \(-a\in I\), so \(I\) contains both positive and negative integers.
   Since \(I\cap \Z^+\neq \varnothing\), the Well Ordering Principle (W.O.P.) implies there is a smallest positive integer in \(I\).
    Take \(d\) as this  integer.
    By \ref{lab:iii.exercise.ideal.of.Z}, we have \((d)\subseteq I\).
    Let's see the other inclusion.
    If \(a\in I\), then by the division algorithm, \(a=qd+r\) for some \(q,r\in \Z\) with \(0\leq r < d\).
    By \ref{lab:ii.exercise.ideal.of.Z}, \(r=a-qd\in I\).
    However, \(d\)  is the smallest positive integer contained in \(I\).
    Since \(0\leq r <d\), the only possibility for this inequality to be true is when  \(r=0\).
    Therefore \(a=qd\).
    It follows \(I =  (d)\), and the proof is complete.
\end{proof}


\begin{teo}[Euclid's lemma]
    Let \(a,b\in \Z\).
    If \(p\) is prime and \(p\mid ab\), then \(p\mid a\) or \(p\mid b\).
\end{teo}

\begin{proof}
    Suppose \(p\) is prime and \(p\mid ab\). We have to prove that \(p\mid a\) or \(p\mid b\). However, this is equivalent to \[p\nmid a\implies p\mid b.\]
    Thus, suppose also \(p\nmid a\).
    Then \((p,a) = 1\) by Remark \ref{rmk:gcd.of.prime.and.integer}.
    By the division algorithm, there are \(x,y\in\Z\) such that \(1=xp+ya\), so \(b=xpb + yab\).
    Because \(p\mid ab\), there is \(c\in \Z\) such that \(ab=cp\).
    Thus \(b=xpb + ycp = (xb+yc)p\), i.e., \(b\) is a multiple of \(p\).
    In other words \(p\mid b\), as desired. 
    The proof is complete.
\end{proof}

\begin{cor}
    Let \(a\in \Z\).
    If \(p\) is  prime and  \(p\mid a^n\) for some \(n\in\Z^+\), then \(p\mid a\).
\end{cor}

\begin{eje}
    Let \(a_1 a_2 \cdots a_n\in \Z\).
    Prove, by induction, that if \(p\) is prime and \(p\mid a_1 a_2 \cdots a_n\), then there is \(i\in \{1,\ldots,n\}\) such that \(p\mid a_i\), i.e., \(p\) must divide at least one integer in the product.
\end{eje}

The converse of Euclid's lemma is also true.

\begin{prop}
    Let \(p>1\).
    Suppose
    \[\forall a,b\in \Z  :\quad p\mid ab\implies p\mid a \text{  or  } p\mid b.\]
    Then \(p\) is prime.
\end{prop}

\begin{proof}
    Assume, for the sake of contradiction, that \(p\) is not prime.
    Then \(p\) is composite, which means \(p=ab\) for some \(a,b\in \left\{ 2,\ldots, p-1 \right\}\).
    Since \(p\mid ab\), the hypothesis implies \(p\mid a\) or \(p\mid b\).
    However, both cases are impossible because \(p\) is greater than \(a\) and \(b\).
    This contradiction proves \(p\) is prime.
\end{proof}


\begin{prop}
    Let \(a,b,c\in \Z\).
    Suppose
    \begin{enumerate}[label=(\roman*)]
        \item \((a,c) = 1\), and
        \item \(c\mid ab\).
    \end{enumerate}
    Then \(c\mid b\).
\end{prop}


\begin{proof}
    By (ii), \(ab=cd\) for some \(d\in\Z\).
    Using (i), write \(1 = ax+cy\) for some \(x,y\in \Z\). Multiplying by \(b\) we get 
    \[b = abx + cby = cdx + cby = (dx + by)c.\]
    Thus \(c\mid b\).
\end{proof}



\begin{defi}
    Let \(a,b\in \Z\) with \(b\neq 0\). We say \(\frac{a}{b}    \) is in lowest terms if \((a,b) = 1\).
\end{defi}

\begin{lem}
    Every nonzero rational number equals a fraction in lowest terms.
\end{lem}

\begin{proof}
    
\end{proof}

\begin{prop}
    \(\sqrt{2}\) is irrational.
\end{prop}

\begin{proof}
    
\end{proof}


\begin{teo}[Fundamental Theorem of Arithmetic]
   For every integer $n>1$, there are unique distinct primes $p_1, \ldots, p_k $ and unique positive integers $a_1, \ldots, a_k $ such that
\[
n=p_1^{a_1} p_2^{a_2} \cdots p_k^{a_k}.
\] 
Moreover, this factorization is unique up to reordering. That is, the product  can be rearranged in any  different order, but these primes and their powers are always the same.
\end{teo}

\begin{proof}
\begin{enumerate}[left=2cm]
    \item[(Existence)] By (strong) induction on $n$. Define
\[
A=\{n \in \mathbb{N}: P(n) \text { is true }\} .
\]

For $n=2$, it is clear that $n \in A$ as $n=2^1$ and 2 is prime. Now, fix $m \in \mathbb{N}$ and assume that $2, \ldots, m \in A$. Lets see that $m+1 \in A$. There are two cases. Either $m+1$ is prime or it is not.
If $m+1$ is prime, there is nothing to show.
If $m+1$ is not prime, then it is composite. Thus, there are $m_1, m_2 \in \mathbb{N}$ such that $m+1=m_1 m_2$. But, given that $m_1, m_2 \in\{2, \ldots, m\}$, it follows that $m_1, m_2 \in A$, by the (inductive) hypothesis. Hence $m+1$ can be expressed as the product of prime numbers, i.e., $m+1 \in A$.

By the principle of mathematical induction, $n \in A$ for every $n \geq 2$.

\item[(Uniqueness)] Suppose that a natural number $n \geq 2$ has two distinct prime decompositions, e.g.,
\[
p_1^{a_1} \cdots p_k^{a_k}=n=q_1^{a_1} \cdots q_l^{a_1} .
\]
where $p_i, q_j \in \mathbb{P}$ and $a_i, b_j \in \mathbb{N}^*$ for $i \in I:=\{1, \ldots, k\}$ and $j \in J:=\{1, \ldots, l\}$. We prove first that the prime numbers on the left are the same as those on the right side of (1). Let $i_0 \in I$. Since $p_{i_0}$ appears at least once in $p_1^{a_1} \cdots p_k^{a_k}$, we have that $p_{b_6} \mid p_1^{a_1} \cdots p_k^{a_k}$ Thus,
\[
p_{i_0} \mid q_1^{a_1} \cdots q_1^{a_1} .
\]

This means that for some $j_0 \in J, q_{j_0}^{b_{j_0}}$ is multiple of $p_{i_0}$. So, $p_{i_0} \mid q_{j_0}^{b_{10}}$.

By Lemma $1, p_{i 0} \mid q_{j 0}$. But $q_{j 0}$ is prime, so it can only be divided by 1 or by itself. Since $p_{i_0} \neq 1, p_{i_0}=q_{i_0}$. As $i_0 \in I$ was arbitrary, we deduce that
\[
\forall i \in I, \exists j_i \in J: p_i=q_i \text {. }
\]

Conversely, let $j_0 \in J$. Since $q_{j 0}$ appears at least once in $q_1^{b_1} \cdots q_l^{b_1}$, we have that $q_{j 0} \mid q_1^{b_1} \cdots q_l^{b_1}$. Thus,
\[
q_{j 0} \mid p_1^{a_1} \cdots p_k^{a_k} .
\]

This means that for some $i_0 \in I, p_{i 0}^{a_{i 0}}$ is multiple of $q_{j 0}$. So, $q_{j 0} \mid p_{i 0}^{a_{i 0}}$. By Lemma $1, q_{j b} \mid p_{i b}$. By the same reasoning as above,
\[
\forall j \in J, \exists i_j \in I: q_j=p_{i j} .
\]
Both (2) and (3) imply that $k=I$, which implies $I=J$, and further that
\[
p_1^{a_1} \cdots p_k^{a_k}=p_1^{b_1} \cdots p_k^{b_k} .
\]

Let's see finally that the corresponding exponents are equal. Suppose, f.s.c., that there is $i \in I$ such that $a_i \neq b_i$. Thus, either $a_i<b_i$ or $b_i<a_i$.
(I) If $a_i<b_i$, then $p_i^{a_i}<p_i^{b_i}$. So, $p_i^{a_i-b_i}<1$ and further
\[
p_i^{b_i} \nmid p_1^{a_1} \cdots p_k^{a_k}, \quad p_i^{b_i} \mid p_1^{b_1} \cdots p_k^{b_k} .
\]

This is a contradiction to (4).

(II) If $b_i<a_i$, then $p_i^{b_i}<p_i^{a_i}$. So, $p_i^{b_i-a_i}<1$ and further
\[
p_i^{a_i} \mid p_1^{a_1} \cdots p_k^{a_k}, \quad p_i^{a_i} \nmid p_1^{b_1} \cdots p_k^{b_k} .
\]

This is a contradiction to (4). (Roughly speaking, when diving the right hand side by $p_i^{a_i}$, there aren't enough $p_i^{b_i}$ to cancell.)
In any case we get a contradiction. Thereby, the assumption is false. That is, $a_i=b_i$ for every $i \in I$. Therefore, both decompositions are, in fact, the same. We have proved that every natural number $n \geq 2$ has a unique prime factorization.
\end{enumerate}
\end{proof}

The following function  computes the amount of smaller integers that are coprime to a given integer.
\begin{defi}[Euler's totient function \(\varphi\)]
    Define \(\varphi \colon \Z^+\to \Z\) by 
    \[\varphi(n) = |\left\{ a\leq n : (a,n) = 1 \right\}|.\]
\end{defi}

\paragraph{Properties.} 

\end{document}
