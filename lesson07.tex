\documentclass[11pt,a4paper]{article}

% PAQUETES
\usepackage[T1]{fontenc}%
\usepackage[utf8]{inputenc}%

\usepackage[english]{babel}
\usepackage{amsmath}
\usepackage{amsthm}
\usepackage{amsfonts}
\usepackage[%left=1.54cm,right=1.54cm,top=1.54cm,bottom=1.54cm
    margin=1in, includefoot,
]{geometry}
\usepackage{xfrac}  
\usepackage{tikz-cd}
\usepackage{enumerate}
\usepackage{amsfonts}
\usepackage{amssymb}
\usepackage{tcolorbox}
\usepackage{rotating}
\usepackage{mathpazo}
% \usepackage{charter}
\usetikzlibrary{babel}
\usepackage{listings}
\usepackage{amssymb}
\usepackage{extarrows}
\usepackage{makeidx}
\usepackage{graphicx}
\usepackage{multirow}
\usepackage{tikz-cd}
\usepackage{tasks}
\usepackage{xcolor}
\usepackage{mathrsfs} % 2024-11-04

% Christian
\usepackage{enumitem,etoolbox,titlesec}


%OPERADORES

\DeclareMathOperator{\dom}{dom}
\DeclareMathOperator{\cod}{cod}
\DeclareMathOperator{\id}{id}

\newcommand{\red}[1]{\textcolor{red}{#1}}
\renewcommand{\ker}{\operatorname{Ker}}
\newcommand{\im}{\operatorname{Im}}
\newcommand{\C}{\mathbb{C}}
\newcommand{\R}{\mathbb{R}}
\newcommand{\Q}{\mathbb{Q}}
\newcommand{\N}{\mathbb{N}}
\newcommand{\Z}{\mathbb{Z}}
\newcommand{\D}{\mathbb{D}}
\newcommand{\B}{\mbox{Ob}}
\newcommand{\M}{\mbox{Mo}}
\newcommand{\del}{\Delta}
\newcommand{\odel}[1]{\left[#1\right]}
\newcommand{\Hom}[1]{\text{Hom}(#1)}
\newcommand{\adel}[1]{\left\lbrace #1 \right\rbrace}
%\renewcommand{\theequation}{\thesection.\arabic{equation}}
\newcommand{\funcion}[5]{%
{\setlength{\arraycolsep}{2pt}
\begin{array}{r@{}ccl}
#1\colon \hspace{0pt}& #2 & \longrightarrow & #3\\
& #4 & \longmapsto & #5
\end{array}}}

\newcommand{\func}[3]{#1\colon  #2  \to  #3}

\newcommand\restr[2]{{% we make the whole thing an ordinary symbol
  \left.\kern-\nulldelimiterspace % automatically resize the bar with \right
  #1 % the function
  \vphantom{\big|} % pretend it's a little taller at normal size
  \right|_{#2} % this is the delimiter
  }}

%ENTORNOS

% \theoremstyle{theorem}
\newtheorem{teo}{Theorem}[section]
\newtheorem{prop}[teo]{Proposition}
\newtheorem{lem}[teo]{Lemma}
\newtheorem{cor}[teo]{Corollary}

\theoremstyle{definition}
\newtheorem{defi}[teo]{Definition}
\newtheorem{rem}{Remark}[teo]
\newtheorem{exa}{Example}
\newtheorem{eje}{Exercise}
\newtheorem{que}{Question}

\newenvironment{sol}
  {\begin{proof}[\textit{Solution}]}
  {\end{proof}}

\titlelabel{\thetitle.\quad}

\def\contador{}
\graphicspath{{./figures/}}
\newcommand{\qand}{\quad\text{and}\quad}
\usepackage{microtype,parskip}
\setlength{\parindent}{15pt}
\linespread{1.15}
\usepackage{hyperref}
\hypersetup{
    colorlinks=true,  
    allcolors=blue,
    pdfproducer={Christian Chávez},
}

\makeatletter
\@ifclassloaded{exam}{
    \footer{}{\thepage}{}
    \renewcommand{\thequestion}{\bfseries\arabic{question}}
    \renewcommand{\solutiontitle}{\noindent\textit{Solution.}\enspace}
    \unframedsolutions
}{}
\makeatother


\newenvironment{theproof}
{
    \renewcommand{\solutiontitle}{}
    \begin{solution}
    \vspace*{-\baselineskip}
    \begin{proof}
}
{
    \end{proof}
    \end{solution}
    \renewcommand{\solutiontitle}{\noindent\textit{Solution.} }
}

\usepackage[style=numeric]{biblatex}
\addbibresource{bibliography.bib}

\begin{document}

\def\contador{Lesson 7}
\noindent
\begin{minipage}[c]{0.33\textwidth}
    \includegraphics[scale=0.37]{sello_yachay.png}
\end{minipage}
\begin{minipage}[c]{0.37\textwidth}
    % \centering
    \textbf{\large School of Mathematical and\\ Computational Sciences}\par
    Abstract Algebra
\end{minipage}
~ 
\begin{minipage}[c]{3mm}
    \raggedleft
    \rule[1.5mm]{0.3mm}{15mm}
\end{minipage}
~ 
\begin{minipage}[c]{0.24\textwidth}
    \raggedleft
    Prof. Pablo Rosero\\
    \& Christian Chávez\\
    \contador{}
\end{minipage}

\vspace{1mm}
\noindent\hrulefill

\vspace{3mm}

\section{Cyclic groups and subgroups}

\begin{defi}
    A group $H$ is cyclic if $H$ can be generated by a single element, i.e., there exists $a \in H$ such that
\[
H = \langle a \rangle = \{a^n \mid n \in \mathbb{Z}\} \text{ where } a^n \in H.
\]
\end{defi}


\begin{rem}
    \begin{enumerate}
        \item In additive notation $H = \{2m \mid m \in \mathbb{Z}\}$. \textit{(In additive notation $(\mathbb{Z}/n\mathbb{Z})$ is cyclic and $\mathbb{Z}/2\mathbb{Z} = \langle 1 \rangle$)}
        \item If $H$ is cyclic then there exists some $x \in H$ such that $H = \langle x \rangle$.
        \item If $\lvert H \rvert = \langle x \rangle$ then $x$ is not unique (and more).
        \item $x^n \neq x^m$ if and only if $n \neq m$.
        \item If $G = D_n$ and $H = \langle r \rangle$, then $H = \langle r^m \rangle$ and $k = m$ if and only if $k \equiv m \mod n$.
        \item      Every cyclic subgroup $H$ is abelian. For example, if $H = \langle r \rangle$ in $G = D_n$, then $H$ is abelian, but $D_n$ is not cyclic.
        \item By convention, \(x^0 = 1\) for any element \(x\)
    \end{enumerate}
\end{rem}


\begin{prop}
    If \( H = \langle x \rangle \) then \( \lvert H \rvert = \lvert x \rvert \). More specifically:
    \begin{enumerate}
        \item If \( \lvert H \rvert = n < \infty \), then \( x^n = 1 \) and \( 1, x, \dots, x^{n-1} \) are all distinct elements of \( H \).
        \item If \( \lvert H \rvert = \infty \), then \( x^n \neq 1 \) for \( n \neq 0 \) and \( x^a \neq x^b \) for \( a \neq b \) in \( \mathbb{Z} \).
    \end{enumerate}
\end{prop}


\begin{prop}
    Let \( G \) be a group, \( x \in G \), and \( m, n \in \mathbb{Z} \backslash \{0\} \).
\begin{itemize}
    \item If \( x^m = 1 \) and \( x^n = 1 \), then \( x^d = 1 \) where \( d = \gcd(m, n) \).
    \item In particular, if \( x^m = 1 \), then \( x^{\lvert m \rvert} = 1 \).
\end{itemize}
\end{prop}



\begin{proof}
    By the Euclidean Algorithm, there exist \( r, s \in \mathbb{Z} \) such that \( d = mr + ns \) where \( d = \gcd(m, n) \).
    Therefore, \( x^d = (x^m)^r \cdot (x^n)^s = 1^r \cdot 1^s = 1 \).

    On the other hand, if \( x^m = 1 \) and \( n = \lvert x \rvert \), then if \( m = 0 \) (implying \( n \mid m \)),  then by 1), \( x^d = 1 \) where \( d = \gcd(m, n) \),
    
    therefore \( d = n \) by minimality. Then (since \( d \mid n \) and \( n \mid m \)), \( d = m \).
\end{proof}


\begin{proof}
    
\end{proof}

\begin{teo}
    Any two cyclic groups of the same order are isomorphic.
\end{teo}

\begin{proof}
    
\begin{enumerate}
    \item[(1)] \textbf{Finite case:}
    Let \( H_1 = \langle x \rangle \) and \( H_2 = \langle y \rangle \) where \( |x| = |y| = n \).
    Define \( \varphi: \langle x \rangle \to \langle y \rangle \) by \( \varphi(x^k) = y^k \).
    Then \( \varphi \) is a well-defined isomorphism.
    
    \begin{itemize}
        \item \textbf{Well-defined:} If \( x^k = x^l \) then \( \varphi(x^k) = \varphi(x^l) \) since \( y^k = y^l \).
        Since \( x^k = x^l \) implies \( k \equiv l \mod n \), \( y^k = y^l \) by the same logic.
        \item \textbf{Homomorphism:} \( \varphi(x^k \cdot x^l) = \varphi(x^{k+l}) = y^{k+l} = y^k \cdot y^l = \varphi(x^k) \cdot \varphi(x^l) \).
        \item \textbf{Injective:} If \( \varphi(x^k) = y^k = 1 \), then \( x^k = 1 \) since \( n \mid k \).
        \item \textbf{Surjective:} Let \( y^k \in \langle y \rangle \) then \( \varphi(x^k) = y^k \).
    \end{itemize}
    
    \item[(2)] \textbf{Infinite case:}
    If \( H = \langle x \rangle \) with \( |H| = \infty \), then define \( \varphi: \mathbb{Z} \to \langle x \rangle \) by \( \varphi(k) = x^k \).
    \( \varphi \) is an isomorphism:
    \begin{itemize}
        \item \( \varphi \) is a function from \( \mathbb{Z} \) to \( \langle x \rangle \) that maps each integer \( k \) to \( x^k \), preserving the structure of \( \mathbb{Z} \) under addition, mirroring the group operation of \( \langle x \rangle \) under multiplication.
    \end{itemize}
\end{enumerate}
\end{proof}

\begin{rem}
    Up to isomorphism, there exists a unique cyclic group of finite order \( n \), namely \( \mathbb{Z}/n\mathbb{Z} = \langle x \rangle = \{1, x, x^2, \ldots, x^{n-1}\} \) (multiplicative), and a unique cyclic group of infinite order, \( \mathbb{Z} = \langle x \rangle = \{n \cdot 1 \mid n \in \mathbb{Z}\} \) (additive).
\end{rem}


\begin{prop}
    Let \( G \) be a group, let \( x \in G \), and let \( a \in \mathbb{Z}\setminus\{0\} \).
\begin{enumerate}[label=(\roman*)]
    \item If \( |x| = \infty \), then \( |x^a| = \infty \).
    \item If \( |x| = n < \infty \), then \( |x^a| = \frac{n}{\gcd(n, a)} \).
    \item If \( |x| = n < \infty \) and also \( a \equiv 0 \mod n \), then \( |x^a| = \frac{n}{a} \).
\end{enumerate}
\end{prop}


\begin{proof}
    \begin{enumerate}
        \item Assume that \( |x| = \infty \). Just assume \( |x^a| = m < \infty \). Then \( (x^a)^m = x^{am} = 1 \). Show that there exist \( r, s \in \mathbb{Z} \) such that \( n = amr + s \) where \( x^n = x^s \). This shows \( |x| < \infty \), which is a contradiction.
        
        \item Define \( y = x^a \) and \( d = \gcd(n, a) \), then \( n = db \) and \( a = dc \) for some \( b, c \in \mathbb{Z} \) with \( \gcd(b, c) = 1 \). We need to prove that \( |y| = b \). First note that \( y^b = (x^a)^b = x^{ab} = x^{dcb} = (x^n)^c = 1^c = 1 \). Thus \( |y| \leq b \).
        
        Let \( k = |y| \), then \( y^k = x^{ak} = 1 \). If \( ak = nd \), since \( \gcd(b, c) = 1 \), then \( b \mid k \). Thus \( k = b \) and hence \( |y| = b \).
        
        \item This is a special case of 2.
    \end{enumerate}
\end{proof}


\begin{teo}
    Let \( H \) be a cyclic group. Assume \( H = \langle x \rangle \).

\begin{enumerate}
    \item Every subgroup \( K \leq H \) is cyclic and \( K = \langle x^d \rangle \) where \( d = \min \{k \in \mathbb{N} \mid x^k \in K \} \).
    
    \item If \( |H| = \infty \), then \( \langle x^s \rangle \neq \langle x^t \rangle \) for all \( s \neq t \) in \( \mathbb{Z} \), and \( \langle x^n \rangle = \langle x \rangle \) implies \( \mathbb{Z} \). Thus, there exists an injective correspondence between \( \mathbb{N} \) and the subgroups of \( H \).
    
    \item If \( |H| = n < \infty \), then for all \( a \in \mathbb{Z}^* \) such that \( a \mid n \) and \( a \neq n \), \( \langle x^d \rangle \leq H \) implies that \( |K| = a \) where \( d \cdot m = n/a \).
    \begin{enumerate}
        \item \( \langle x^s \rangle = \langle x^{(n/m)} \rangle \) where \( \gcd(m, n) = 1 \).
    \end{enumerate}
    
    \item The subgroups of \( H \) correspond bijectively with the positive divisors of \( |H| \).
\end{enumerate}
\end{teo}

\begin{rem}
    In \( \mathbb{Z}/n\mathbb{Z} \):
    \begin{enumerate}
        \item \( \mathbb{Z}/n\mathbb{Z} = \langle t \rangle = \langle m \rangle \) if and only if \( \gcd(m, n) = 1 \) for \( m \in \mathbb{Z} \).
        \item \( \langle s \rangle \leq \langle \gcd(s, m) \rangle \).
        \item \( \langle a \rangle \leq \langle b \rangle \) if and only if \( \gcd(b, n) \mid \gcd(a, n) \) where \( 1 \leq a, b \leq n \).
    \end{enumerate}
\end{rem}

\begin{exa}
    In \( \mathbb{Z}/48\mathbb{Z} \), compute \( \langle 6 \rangle \), find the order of \( a \) and relation between \( \langle 6 \rangle \) and Molien subgroups.
\begin{itemize}
    \item \( \phi(48) = \phi(2^4 \cdot 3) = \phi(2^4) \cdot \phi(3) = 2^3 \cdot (3-1) = 16 \).
\end{itemize}
\end{exa}


The subgroup relations for \( \mathbb{Z}/48\mathbb{Z} \) are represented as follows:

\begin{align*}
    \langle 1 \rangle & = \langle 47 \rangle = \langle 49 \rangle = \cdots = \langle 1 \rangle, \\
    \langle 2 \rangle & = \langle 46 \rangle = \langle 50 \rangle = \cdots = \langle 2 \rangle, \\
    \langle 3 \rangle & = \langle 45 \rangle = \langle 51 \rangle = \cdots = \langle 3 \rangle, \\
    \langle 4 \rangle & = \langle 44 \rangle = \langle 52 \rangle = \cdots = \langle 4 \rangle, \\
    \langle 6 \rangle & = \langle 42 \rangle = \langle 54 \rangle = \cdots = \langle 6 \rangle, \\
    \langle 8 \rangle & = \langle 40 \rangle = \langle 56 \rangle = \cdots = \langle 8 \rangle, \\
    \langle 12 \rangle & = \langle 36 \rangle = \langle 60 \rangle = \cdots = \langle 12 \rangle, \\
    \langle 16 \rangle & = \langle 32 \rangle = \langle 64 \rangle = \cdots = \langle 16 \rangle, \\
    \langle 24 \rangle & = \langle 24 \rangle = \langle 72 \rangle = \cdots = \langle 24 \rangle.
\end{align*}

Subgroups of \( \mathbb{Z}/48\mathbb{Z} \) are related as follows:
\begin{align*}
    \langle 24 \rangle & \subset \langle 12 \rangle \subset \langle 6 \rangle \subset \langle 3 \rangle \subset \langle 1 \rangle, \\
    \langle 16 \rangle & \subset \langle 8 \rangle \subset \langle 4 \rangle \subset \langle 2 \rangle \subset \langle 1 \rangle, \\
    \langle 18 \rangle & \subset \langle 9 \rangle \subset \langle 3 \rangle \subset \langle 1 \rangle, \\
    \langle 20 \rangle & \subset \langle 10 \rangle \subset \langle 5 \rangle \subset \langle 1 \rangle.
\end{align*}

\end{document}