\documentclass[11pt,a4paper]{article}

% PAQUETES
\usepackage[T1]{fontenc}%
\usepackage[utf8]{inputenc}%

\usepackage[english]{babel}
\usepackage{amsmath}
\usepackage{amsthm}
\usepackage{amsfonts}
\usepackage[%left=1.54cm,right=1.54cm,top=1.54cm,bottom=1.54cm
    margin=1in, includefoot,
]{geometry}
\usepackage{xfrac}  
\usepackage{tikz-cd}
\usepackage{enumerate}
\usepackage{amsfonts}
\usepackage{amssymb}
\usepackage{tcolorbox}
\usepackage{rotating}
\usepackage{mathpazo}
% \usepackage{charter}
\usetikzlibrary{babel}
\usepackage{listings}
\usepackage{amssymb}
\usepackage{extarrows}
\usepackage{makeidx}
\usepackage{graphicx}
\usepackage{multirow}
\usepackage{tikz-cd}
\usepackage{tasks}
\usepackage{xcolor}
\usepackage{mathrsfs} % 2024-11-04

% Christian
\usepackage{enumitem,etoolbox,titlesec}


%OPERADORES

\DeclareMathOperator{\dom}{dom}
\DeclareMathOperator{\cod}{cod}
\DeclareMathOperator{\id}{id}

\newcommand{\red}[1]{\textcolor{red}{#1}}
\renewcommand{\ker}{\operatorname{Ker}}
\newcommand{\im}{\operatorname{Im}}
\newcommand{\C}{\mathbb{C}}
\newcommand{\R}{\mathbb{R}}
\newcommand{\Q}{\mathbb{Q}}
\newcommand{\N}{\mathbb{N}}
\newcommand{\Z}{\mathbb{Z}}
\newcommand{\D}{\mathbb{D}}
\newcommand{\B}{\mbox{Ob}}
\newcommand{\M}{\mbox{Mo}}
\newcommand{\del}{\Delta}
\newcommand{\odel}[1]{\left[#1\right]}
\newcommand{\Hom}[1]{\text{Hom}(#1)}
\newcommand{\adel}[1]{\left\lbrace #1 \right\rbrace}
%\renewcommand{\theequation}{\thesection.\arabic{equation}}
\newcommand{\funcion}[5]{%
{\setlength{\arraycolsep}{2pt}
\begin{array}{r@{}ccl}
#1\colon \hspace{0pt}& #2 & \longrightarrow & #3\\
& #4 & \longmapsto & #5
\end{array}}}

\newcommand{\func}[3]{#1\colon  #2  \to  #3}

\newcommand\restr[2]{{% we make the whole thing an ordinary symbol
  \left.\kern-\nulldelimiterspace % automatically resize the bar with \right
  #1 % the function
  \vphantom{\big|} % pretend it's a little taller at normal size
  \right|_{#2} % this is the delimiter
  }}

%ENTORNOS

% \theoremstyle{theorem}
\newtheorem{teo}{Theorem}[section]
\newtheorem{prop}[teo]{Proposition}
\newtheorem{lem}[teo]{Lemma}
\newtheorem{cor}[teo]{Corollary}

\theoremstyle{definition}
\newtheorem{defi}[teo]{Definition}
\newtheorem{rem}{Remark}[teo]
\newtheorem{exa}{Example}
\newtheorem{eje}{Exercise}
\newtheorem{que}{Question}

\newenvironment{sol}
  {\begin{proof}[\textit{Solution}]}
  {\end{proof}}

\titlelabel{\thetitle.\quad}

\def\contador{}
\graphicspath{{./figures/}}
\newcommand{\qand}{\quad\text{and}\quad}
\usepackage{microtype,parskip}
\setlength{\parindent}{15pt}
\linespread{1.15}
\usepackage{hyperref}
\hypersetup{
    colorlinks=true,  
    allcolors=blue,
    pdfproducer={Christian Chávez},
}

\makeatletter
\@ifclassloaded{exam}{
    \footer{}{\thepage}{}
    \renewcommand{\thequestion}{\bfseries\arabic{question}}
    \renewcommand{\solutiontitle}{\noindent\textit{Solution.}\enspace}
    \unframedsolutions
}{}
\makeatother


\newenvironment{theproof}
{
    \renewcommand{\solutiontitle}{}
    \begin{solution}
    \vspace*{-\baselineskip}
    \begin{proof}
}
{
    \end{proof}
    \end{solution}
    \renewcommand{\solutiontitle}{\noindent\textit{Solution.} }
}

\usepackage[style=numeric]{biblatex}
\addbibresource{bibliography.bib}

\begin{document}

\def\contador{Lesson 14}
\noindent
\begin{minipage}[c]{0.33\textwidth}
    \includegraphics[scale=0.37]{sello_yachay.png}
\end{minipage}
\begin{minipage}[c]{0.37\textwidth}
    % \centering
    \textbf{\large School of Mathematical and\\ Computational Sciences}\par
    Abstract Algebra
\end{minipage}
~ 
\begin{minipage}[c]{3mm}
    \raggedleft
    \rule[1.5mm]{0.3mm}{15mm}
\end{minipage}
~ 
\begin{minipage}[c]{0.24\textwidth}
    \raggedleft
    Prof. Pablo Rosero\\
    \& Christian Chávez\\
    \contador{}
\end{minipage}

\vspace{1mm}
\noindent\hrulefill

\vspace{3mm}
 

\section{Polynomial Rings and UFDs}

We have seen that if $A$ is an integral domain, then $A[x]$ is also an integral domain. If $Q$ is the field of fractions of $A$, then $A[x] \subseteq Q[x]$, and $Q[x]$ is an Euclidean Domain, a PID, and a UFD. Then all polynomials in $A[x]$ can be uniquely factored over $Q[x]$.

Therefore, we want to know how a factorization in $Q[x]$ can help us to factor over $A[x]$ although $A[x]$ is not always a UFD. For this, we shall need the famous Gauss's Lemma.

\begin{proposition}[10.6]
Let $I$ be an ideal of the ring $A$ and let $I[x]$ denote the ideal of $A[x]$ generated by $I$, i.e., the set of polynomials with coefficients in $I$. Then,
\[
\frac{A[x]}{I[x]} \cong \left(\frac{A}{I}\right)[x].
\]
\end{proposition}

\begin{proof}
Let's define the surjective ring homomorphism
\[
\theta : A[x] \to \left(\frac{A}{I}\right)[x]
\]
by reducing each of the coefficients of a polynomial modulo $I$. It is clear that the kernel of $\theta$ is the set of polynomials each of whose coefficients is an element of $I$, i.e.,
\[
\ker(\theta) = I[x].
\]
Then, by the first theorem of isomorphism, we have that
\[
\frac{A[x]}{I[x]} \cong \left(\frac{A}{I}\right)[x].
\]
\end{proof}

\begin{remark}[10.3]
Proposition 10.6 implies that if $I$ is a prime ideal of $A$, then $I[x]$ is a prime ideal of $A[x]$.
\end{remark}

\begin{theorem}[10.2 (Gauss's Lemma)]
Let $A$ be a UFD and $Q$ the field of fractions of $A$. If $p(x)$ is reducible in $Q[x]$, then $p(x)$ is reducible in $A[x]$. Moreover, if $p(x) = r(x)s(x)$ for some non-constant polynomials $r(x), s(x) \in Q[x]$, then there are nonzero elements $A, B \in Q$ such that $Ar(x) = a(x)$ and $Bs(x) = b(x)$ and
\[
a(x) \in A[x], \quad b(x) \in A[x], \quad p(x) = a(x)b(x).
\]
Therefore, $a(x)b(x)$ is a factorization of $p(x)$ in $A[x]$.
\end{theorem}

\begin{proof}
In the equality $p(x) = r(x)s(x)$, the coefficients of the term $r(x)s(x)$ are elements of $Q$ by hypothesis. Then, it is possible to obtain the equality
\[
dp(x) = a'(x)b'(x),
\]
where $d$ represents the common denominator of all the coefficients of $r(x)s(x)$ and $a'(x), b'(x) \in A[x]$.

\begin{enumerate}[i)]
    \item If $d$ is invertible, then take $a(x) = d^{-1}a'(x)$ and $b(x) = d^{-1}b'(x)$ and the proof is complete.

    \item If $d$ is not invertible, since $A$ is a UFD and $d = p_1 \cdots p_n$, it follows that $p_1$ is irreducible and $\langle p_1 \rangle$ is a prime ideal. Therefore, by Proposition 10.6, the ring $(A/p_1A)[x]$ is an integral domain and $p_1A[x]$ is prime in $A[x]$. Reducing modulo $p_1$ over the quotient ring $(A/p_1A)[x]$, the equality $dp(x) = a'(x)b'(x)$ becomes
    \[
    0 = a'(x)b'(x),
    \]
    where the bars denote the equivalence class in this quotient ring. Since this ring is an integral domain, one of the factors must be $0$. Say, $a'(x) = 0$. Therefore, all the coefficients of $a'(x)$ are divided by $p_1$, so $\frac{1}{p_1}a'(x) \in A[x]$. Thus we can simplify the factor $p_1$ from the factorization of $d$ in the equality $dp(x) = a'(x)b'(x)$. Proceeding in the same way with each of the remaining factors of $d$, we can cancel $d$ in the equation $dp(x) = a'(x)b'(x)$ and obtain a factorization
    \[
    p(x) = a(x)b(x),
    \]
    where $a(x), b(x) \in A[x]$ are multiples of $r(x)$ and $s(x)$ by elements of $Q$, respectively.
\end{enumerate}
\end{proof}

\begin{corollary}[10.2]
Let $A$ be a UFD and $Q$ be its field of fractions, and let $p(x) \in A[x]$. If the greatest common divisor of $p(x)$ is $1$, then $p(x)$ is irreducible in $A[x]$ if and only if it is irreducible in $Q[x]$. In particular, every monic polynomial that is irreducible in $A[x]$ is also irreducible in $Q[x]$.
\end{corollary}


\end{document}