\documentclass[11pt,a4paper]{article}

% PAQUETES
\usepackage[T1]{fontenc}%
\usepackage[utf8]{inputenc}%

\usepackage[english]{babel}
\usepackage{amsmath}
\usepackage{amsthm}
\usepackage{amsfonts}
\usepackage[%left=1.54cm,right=1.54cm,top=1.54cm,bottom=1.54cm
    margin=1in, includefoot,
]{geometry}
\usepackage{xfrac}  
\usepackage{tikz-cd}
\usepackage{enumerate}
\usepackage{amsfonts}
\usepackage{amssymb}
\usepackage{tcolorbox}
\usepackage{rotating}
\usepackage{mathpazo}
% \usepackage{charter}
\usetikzlibrary{babel}
\usepackage{listings}
\usepackage{amssymb}
\usepackage{extarrows}
\usepackage{makeidx}
\usepackage{graphicx}
\usepackage{multirow}
\usepackage{tikz-cd}
\usepackage{tasks}
\usepackage{xcolor}
\usepackage{mathrsfs} % 2024-11-04

% Christian
\usepackage{enumitem,etoolbox,titlesec}


%OPERADORES

\DeclareMathOperator{\dom}{dom}
\DeclareMathOperator{\cod}{cod}
\DeclareMathOperator{\id}{id}

\newcommand{\red}[1]{\textcolor{red}{#1}}
\renewcommand{\ker}{\operatorname{Ker}}
\newcommand{\im}{\operatorname{Im}}
\newcommand{\C}{\mathbb{C}}
\newcommand{\R}{\mathbb{R}}
\newcommand{\Q}{\mathbb{Q}}
\newcommand{\N}{\mathbb{N}}
\newcommand{\Z}{\mathbb{Z}}
\newcommand{\D}{\mathbb{D}}
\newcommand{\B}{\mbox{Ob}}
\newcommand{\M}{\mbox{Mo}}
\newcommand{\del}{\Delta}
\newcommand{\odel}[1]{\left[#1\right]}
\newcommand{\Hom}[1]{\text{Hom}(#1)}
\newcommand{\adel}[1]{\left\lbrace #1 \right\rbrace}
%\renewcommand{\theequation}{\thesection.\arabic{equation}}
\newcommand{\funcion}[5]{%
{\setlength{\arraycolsep}{2pt}
\begin{array}{r@{}ccl}
#1\colon \hspace{0pt}& #2 & \longrightarrow & #3\\
& #4 & \longmapsto & #5
\end{array}}}

\newcommand{\func}[3]{#1\colon  #2  \to  #3}

\newcommand\restr[2]{{% we make the whole thing an ordinary symbol
  \left.\kern-\nulldelimiterspace % automatically resize the bar with \right
  #1 % the function
  \vphantom{\big|} % pretend it's a little taller at normal size
  \right|_{#2} % this is the delimiter
  }}

%ENTORNOS

% \theoremstyle{theorem}
\newtheorem{teo}{Theorem}[section]
\newtheorem{prop}[teo]{Proposition}
\newtheorem{lem}[teo]{Lemma}
\newtheorem{cor}[teo]{Corollary}

\theoremstyle{definition}
\newtheorem{defi}[teo]{Definition}
\newtheorem{rem}{Remark}[teo]
\newtheorem{exa}{Example}
\newtheorem{eje}{Exercise}
\newtheorem{que}{Question}

\newenvironment{sol}
  {\begin{proof}[\textit{Solution}]}
  {\end{proof}}

\titlelabel{\thetitle.\quad}

\def\contador{}
\graphicspath{{./figures/}}
\newcommand{\qand}{\quad\text{and}\quad}
\usepackage{microtype,parskip}
\setlength{\parindent}{15pt}
\linespread{1.15}
\usepackage{hyperref}
\hypersetup{
    colorlinks=true,  
    allcolors=blue,
    pdfproducer={Christian Chávez},
}

\makeatletter
\@ifclassloaded{exam}{
    \footer{}{\thepage}{}
    \renewcommand{\thequestion}{\bfseries\arabic{question}}
    \renewcommand{\solutiontitle}{\noindent\textit{Solution.}\enspace}
    \unframedsolutions
}{}
\makeatother


\newenvironment{theproof}
{
    \renewcommand{\solutiontitle}{}
    \begin{solution}
    \vspace*{-\baselineskip}
    \begin{proof}
}
{
    \end{proof}
    \end{solution}
    \renewcommand{\solutiontitle}{\noindent\textit{Solution.} }
}

\usepackage[style=numeric]{biblatex}
\addbibresource{bibliography.bib}

\begin{document}

\def\contador{Lesson 6}
\noindent
\begin{minipage}[c]{0.33\textwidth}
    \includegraphics[scale=0.37]{sello_yachay.png}
\end{minipage}
\begin{minipage}[c]{0.37\textwidth}
    % \centering
    \textbf{\large School of Mathematical and\\ Computational Sciences}\par
    Abstract Algebra
\end{minipage}
~ 
\begin{minipage}[c]{3mm}
    \raggedleft
    \rule[1.5mm]{0.3mm}{15mm}
\end{minipage}
~ 
\begin{minipage}[c]{0.24\textwidth}
    \raggedleft
    Prof. Pablo Rosero\\
    \& Christian Chávez\\
    \contador{}
\end{minipage}

\vspace{1mm}
\noindent\hrulefill

\vspace{3mm}

\section{Quotient Groups and Homomorphisms}

\subsection{Cosets and counting}

Let \((G,\cdot)\) be a group.


\begin{defi}
Let \(H\leq G\) and \(a,b\in G\).
Define \(\cong_r\) over \(G\) by 
\[a\cong_r b\iff ab^{-1}\in H.\]
Whenever \(a\cong_r b\) we say \(a\) right is congruent to \(b\) module \(H\).
Define \(\cong_l\) over \(G\) by 
\[a\cong_l b\iff a^{-1}b\in H.\]
Whenever \(a\cong_l b\) we say \(a\) is left congruent to \(b\) module \(H\).
\end{defi}

\begin{rem}
    If \(G\) is Abelian, \[ab^{-1}\in H\iff a^{-1}b\in H\]
    for any \(a,b\in G\).
\end{rem}


\begin{teo}
    Let \(H\leq G\).
    \begin{enumerate}[label=(\roman*)]
        \item Right and left congruence module $H$ are both equivalence relations on $G$.
    \item The equivalence class of $a \in G$ under right  congruence mod $H$ is the set
    \[
    Ha = \{ha \mid h \in H\}  
    \]
    \item The equivalence class of $a \in G$ under left congruence mod $H$ is the set
    \[
     aH = \{ah \mid h \in H\}
    \]

    \item \(\left|H a\right|=|H|=|a H|\) for any \(a\in G\).
    \end{enumerate}
\end{teo}

We call \(aH\) the  left coset of \(H\) by \(a\) in \(G\), and \(Ha\) the a right coset of \(H\) by \(a\) in \(G\). 

\begin{rem}
    In additive notation (that is, when we are working with an Abelian group) we write \(a+H\) instead of \(aH\) and \(H+a\) instead of \(Ha\).
    In fact, there is no difference between left and right cosets in this case. (Why \(a+ H = H + a\) for any \(a\in G\)?)
\end{rem}
 

\begin{proof}
\begin{enumerate}[label=(\roman*)]
    \item 
\end{enumerate}
\end{proof}


\begin{cor}
Let \(H\leq G\).
\begin{enumerate}[label=(\roman*)]
    \item      \( G = \bigcup\limits_{a \in G} Ha = \bigcup\limits_{a \in G} aH \)
    \item For all \(a,b\in G\) distinct, \( aH \cap bH = \emptyset \) and \( Ha \cap Hb = \emptyset \).
    \item For all \(a,b\in G\), we have \( aH = bH \) if and only if \( a^{-1}b \in H \) (or \( b-a \in H \) in additive notation)  and \( Ha = Hb \) if and only if \( ab^{-1} \in H \) (or \( b-a a \in H \) in additive notation).
    \item If \( \mathcal{R} = \{ Ha \mid a \in G \} \) and \( \mathcal{L} = \{ aH \mid a \in G \} \) then \(|\mathcal{R}|=|\mathcal{L}|\).
\end{enumerate}
 
\end{cor}

A special name and notation have been adopted for the number of left (or right) cosets of a subgroup in a group.

\begin{defi}[Index]
    The index of a subgroup $H$ in $G$ is the number of distinct left cosets of $H$ in $G$. This number is denoted by $|G: H|$. 
\end{defi}

\begin{eje}
    Prove $|G: H|$ equals the number of distinct right cosets of $H$ in $G$.
\end{eje}

\begin{teo}
    If $K$, $H$, $G$ are groups  with $k<H<G$ then
\[
[G: k]=[G: H] \cdot[H: k]
\]
If any two of these indices are finite, so is the third.
\end{teo}


\begin{proof}
    
\end{proof}


\begin{cor}[Lagrange's theorem]
    If \( H \leq G \), then \( |G| = [G : H] |H| \). 
In particular, if \( G \) is finite, the order of  any \( a \in G \) divides \( |G| \).
\end{cor}

\begin{proof}
    
\end{proof}


\begin{teo}
    Let \(H\) and \(K\) be finite subgroups of a group \(G\). Then \[|HK| =  \frac{|H||K|}{|H\cap K|}.\]
\end{teo}


\begin{prop}
    
\end{prop}

\begin{proof}
    
\end{proof}


\begin{prop}
    
\end{prop}

\end{document}