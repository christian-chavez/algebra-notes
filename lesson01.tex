\documentclass[11pt,a4paper]{article}

% PAQUETES

\usepackage[spanish]{babel}
\usepackage{amsmath}
\usepackage{amsthm}
\usepackage{amsfonts}
\usepackage[left=1.54cm,right=1.54cm,top=1.54cm,bottom=1.54cm]{geometry}
\usepackage{xfrac}  
\usepackage{tikz-cd}
\usepackage{enumerate}
\usepackage{amsfonts}
\usepackage{amssymb}
\usepackage{tcolorbox}
\usepackage{rotating}
\usepackage{mathpazo}
\usepackage{charter}
\usetikzlibrary{babel}
\usepackage{listings}
\usepackage{amssymb}
\usepackage{extarrows}
\usepackage{makeidx}
\usepackage{graphicx}
\usepackage{multirow}
\usepackage{tikz-cd}
\usepackage{tasks}
\usepackage{color}


%OPERADORES

\DeclareMathOperator{\dom}{dom}
\DeclareMathOperator{\cod}{cod}
\DeclareMathOperator{\id}{id}


\newcommand{\C}{\mathbb{C}}
\newcommand{\im}{\text{Im}}
\newcommand{\R}{\mathbb{R}}
\newcommand{\N}{\mathbb{N}}
\newcommand{\Z}{\mathbb{Z}}
\newcommand{\D}{\mathbb{D}}
\newcommand{\B}{\mbox{Ob}}
\newcommand{\M}{\mbox{Mo}}
\newcommand{\del}{\Delta}
\newcommand{\odel}[1]{\left[#1\right]}
\newcommand{\Hom}[1]{\text{Hom}(#1)}
\newcommand{\adel}[1]{\left\lbrace #1 \right\rbrace}
%\renewcommand{\theequation}{\thesection.\arabic{equation}}
\newcommand{\funcion}[5]{%
{\setlength{\arraycolsep}{2pt}
\begin{array}{r@{}ccl}
#1\colon \hspace{0pt}& #2 & \longrightarrow & #3\\
& #4 & \longmapsto & #5
\end{array}}}

\newcommand{\func}[3]{#1\colon  #2  \to  #3}

\newcommand\restr[2]{{% we make the whole thing an ordinary symbol
  \left.\kern-\nulldelimiterspace % automatically resize the bar with \right
  #1 % the function
  \vphantom{\big|} % pretend it's a little taller at normal size
  \right|_{#2} % this is the delimiter
  }}

%ENTORNOS

\theoremstyle{theorem}
\newtheorem{teo}{Theorem}[section]
\newtheorem{prop}{Proposition}[section]
\newtheorem{lem}{Lemma}[section]
\newtheorem{cor}{Corollary}[section]

\theoremstyle{definition}
\newtheorem{defi}{Definition}[section]
\newtheorem{rem}{Remark}
\newtheorem{eje}{Exercise}
\newtheorem{sol}{\textit{Solution}}
\newtheorem{que}{Question}


\def\contador{}
\graphicspath{{./figures/}}


\begin{document}

\def\contador{1}
\begin{minipage}[c]{30mm}
\includegraphics[scale=0.5]{sello_yachay.png}
\end{minipage}
\begin{minipage}[c]{95mm}
\textbf{School of Mathematical and Computational Sciences}
\end{minipage}
\begin{minipage}[c]{2mm}
\rule[2.5mm]{0.3mm}{15mm}
\end{minipage}
\begin{minipage}[c]{50mm}
Prof. Pablo Rosero.\\ Abstract Algebra: Lesson \contador{}
\end{minipage}

\hrulefill
\vspace{3mm}

\section{Basic properties of the integers}

fi

In this lesson and onwards, we consider $\Z$ to be the set of integers numbers, whereas $\Z^{+}$ is the set of strictly positive integers numbers.

\begin{defi}
Let $a,b\in \Z$, with $a\neq 0,$ then $a$ is a divisor of $b$ if there is an integer $c$ such that $a\cdot c=b$. We denote this by $a\mid b.$ 
\end{defi}

\begin{rem}
    If $a$ does not divide $b$, we write $a\nmid b$.
\end{rem}

\begin{teo}
    Let $a,b\in \Z\setminus\{0\}$, there is a unique positive integer $d$, called the \textbf{greatest common divisor of $a$ and $b$}, satisfying 
    \begin{enumerate}
        \item $d\mid a$ and $d\mid b.$
        \item If $e\mid a$ and $e\mid b$ then $e\mid d.$
    \end{enumerate}
\end{teo}

\begin{rem}
    If $d$ is the greatest common divisor of $a$ and $b$, we write $d=(a,b).$ In particular, if $(a,b)=1$, then $a$ and $b$ are called coprimes.
\end{rem}

\begin{que}
    Why does $(a,b)$ always exist for $a,b\in \Z\setminus\{0\}$?
\end{que}


\end{document}
