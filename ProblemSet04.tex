\documentclass[
    11pt,a4paper,
    % answers
]{exam}

% PAQUETES
\usepackage[T1]{fontenc}%
\usepackage[utf8]{inputenc}%

\usepackage[english]{babel}
\usepackage{amsmath}
\usepackage{amsthm}
\usepackage{amsfonts}
\usepackage[%left=1.54cm,right=1.54cm,top=1.54cm,bottom=1.54cm
    margin=1in, includefoot,
]{geometry}
\usepackage{xfrac}  
\usepackage{tikz-cd}
\usepackage{enumerate}
\usepackage{amsfonts}
\usepackage{amssymb}
\usepackage{tcolorbox}
\usepackage{rotating}
\usepackage{mathpazo}
% \usepackage{charter}
\usetikzlibrary{babel}
\usepackage{listings}
\usepackage{amssymb}
\usepackage{extarrows}
\usepackage{makeidx}
\usepackage{graphicx}
\usepackage{multirow}
\usepackage{tikz-cd}
\usepackage{tasks}
\usepackage{xcolor}
\usepackage{mathrsfs} % 2024-11-04

% Christian
\usepackage{enumitem,etoolbox,titlesec}


%OPERADORES

\DeclareMathOperator{\dom}{dom}
\DeclareMathOperator{\cod}{cod}
\DeclareMathOperator{\id}{id}

\newcommand{\red}[1]{\textcolor{red}{#1}}
\renewcommand{\ker}{\operatorname{Ker}}
\newcommand{\im}{\operatorname{Im}}
\newcommand{\C}{\mathbb{C}}
\newcommand{\R}{\mathbb{R}}
\newcommand{\Q}{\mathbb{Q}}
\newcommand{\N}{\mathbb{N}}
\newcommand{\Z}{\mathbb{Z}}
\newcommand{\D}{\mathbb{D}}
\newcommand{\B}{\mbox{Ob}}
\newcommand{\M}{\mbox{Mo}}
\newcommand{\del}{\Delta}
\newcommand{\odel}[1]{\left[#1\right]}
\newcommand{\Hom}[1]{\text{Hom}(#1)}
\newcommand{\adel}[1]{\left\lbrace #1 \right\rbrace}
%\renewcommand{\theequation}{\thesection.\arabic{equation}}
\newcommand{\funcion}[5]{%
{\setlength{\arraycolsep}{2pt}
\begin{array}{r@{}ccl}
#1\colon \hspace{0pt}& #2 & \longrightarrow & #3\\
& #4 & \longmapsto & #5
\end{array}}}

\newcommand{\func}[3]{#1\colon  #2  \to  #3}

\newcommand\restr[2]{{% we make the whole thing an ordinary symbol
  \left.\kern-\nulldelimiterspace % automatically resize the bar with \right
  #1 % the function
  \vphantom{\big|} % pretend it's a little taller at normal size
  \right|_{#2} % this is the delimiter
  }}

%ENTORNOS

% \theoremstyle{theorem}
\newtheorem{teo}{Theorem}[section]
\newtheorem{prop}[teo]{Proposition}
\newtheorem{lem}[teo]{Lemma}
\newtheorem{cor}[teo]{Corollary}

\theoremstyle{definition}
\newtheorem{defi}[teo]{Definition}
\newtheorem{rem}{Remark}[teo]
\newtheorem{exa}{Example}
\newtheorem{eje}{Exercise}
\newtheorem{que}{Question}

\newenvironment{sol}
  {\begin{proof}[\textit{Solution}]}
  {\end{proof}}

\titlelabel{\thetitle.\quad}

\def\contador{}
\graphicspath{{./figures/}}
\newcommand{\qand}{\quad\text{and}\quad}
\usepackage{microtype,parskip}
\setlength{\parindent}{15pt}
\linespread{1.15}
\usepackage{hyperref}
\hypersetup{
    colorlinks=true,  
    allcolors=blue,
    pdfproducer={Christian Chávez},
}

\makeatletter
\@ifclassloaded{exam}{
    \footer{}{\thepage}{}
    \renewcommand{\thequestion}{\bfseries\arabic{question}}
    \renewcommand{\solutiontitle}{\noindent\textit{Solution.}\enspace}
    \unframedsolutions
}{}
\makeatother


\newenvironment{theproof}
{
    \renewcommand{\solutiontitle}{}
    \begin{solution}
    \vspace*{-\baselineskip}
    \begin{proof}
}
{
    \end{proof}
    \end{solution}
    \renewcommand{\solutiontitle}{\noindent\textit{Solution.} }
}

\usepackage[style=numeric]{biblatex}
\addbibresource{bibliography.bib}

\begin{document}
\def\contador{Problem Set 3}
\noindent
\begin{minipage}[c]{0.33\textwidth}
    \includegraphics[scale=0.37]{sello_yachay.png}
\end{minipage}
\begin{minipage}[c]{0.37\textwidth}
    % \centering
    \textbf{\large School of Mathematical and\\ Computational Sciences}\par
    Abstract Algebra
\end{minipage}
~ 
\begin{minipage}[c]{3mm}
    \raggedleft
    \rule[1.5mm]{0.3mm}{15mm}
\end{minipage}
~ 
\begin{minipage}[c]{0.24\textwidth}
    \raggedleft
    Prof. Pablo Rosero\\
    \& Christian Chávez\\
    \contador{}
\end{minipage}

\vspace{1mm}
\noindent\hrulefill

\vspace{3mm}

\begin{center}
    {\bfseries\Large
        Cyclic Groups, Normality, Quotients  \\ and the Isomorphism Theorems}\\
    (Lessons 6, 7, and 8)
\end{center}


\begin{questions}
    \section{Workout guide}

\begin{ejercicio}
Let $A$ be a ring with unit. Show that if $u$ is invertible in $A$ then so is $-u$.
\end{ejercicio}

\begin{ejercicio}
 Prove that $\entire_p$ is a field iff $p$ is prime. 
 \textbf{Hint:} Prove first that $\entire_p$ is a integral domian iff $p$ is prime, then use Proposition \ref{FID}.
\end{ejercicio}

\begin{ejercicio}
Prove point iii) of \ref{teo:truillo1}.
\end{ejercicio}

%\begin{ejercicio}
%Prove Theorem 
%\ref{teo:truillo2}.
%\end{ejercicio}

\begin{ejercicio}
The \underline{direct product} of the rings $A$ and $B$ is the cartesian product $A\times B$ endowed with the operations defined by
\begin{align*}
(x_1,y_1) + (x_2,y_2) &= (x_1+x_2, y_1+y_2),
\\
(x_1,y_1) \cdot (x_2,y_2) &= (x_1 x_2, y_1 y_2).
\end{align*}
\begin{my_enumerate}
\item Prove that $A\times B$ is a ring with these operations.
\item Prove that if $A$ and $B$ are Abelian, then so is its direct product.
\item Describe the divisors of zero of $A\times B$.
\item Describe the invertible elements of $A\times B$.
\item Assume that $A$ and $B$ are non trivial rings. Explain why $A\times B$ can not be an integral domain.
\end{my_enumerate}
\end{ejercicio}

\begin{ejercicio}
Let $I$ be any nonempty index set and let $\adel{A_i}_{i\in I}$ an indexed family of rings. 
\begin{my_enumerate}
    \item Prove that $\prod_{i\in I} A_i$ is a ring under componentwise addition and multiplication. 
    \item If $I=\entire^{+}\cup\adel{0},$ prove that $\coprod_{i\in I} A_i$  under componentwise addition and multiplication. 
    \item Prove that $A\odel{x}$ is a ring. 
\end{my_enumerate}
\end{ejercicio}

\begin{ejercicio}
Let $(A,+,\cdot)$ be a ring.
\begin{my_enumerate}
\item Prove that for any $a,b,c\in A$,
$$
a(b-c) = ab - ac,
\quad
(b-c)a = ba - ca.
$$
\item Assume that $a,b\in A$ are such that $ab=-ba$. Prove that 
$$
(a+b)^2 = (a-b)^2 = a^2 + b^2.
$$
\item Assume that $A$ is an integral domain. Prove that
\begin{align*}
    \forall a,b\in A&:\quad a^2 = b^2 \: \Rightarrow \: (a=b\, \vee \, a=-b);\\
    \forall x\in A&:\quad
x = x^{-1}\: \Rightarrow \:
x \in \{-1,1\}.
\end{align*}
\item Prove that if $(A,+)$ is a cyclic group, then $A$ is a commutative ring.
\end{my_enumerate}
\end{ejercicio}


\begin{ejercicio}
 Let $A$ a non-void set equipped with internal operations $+$ and $\cdot$. Assume that $(A,+)$ is a group,
 $(A,\cdot)$ is a semigroup, and that
$$
\forall a,b,c \in A:\quad
a \cdot (b+c) = a\cdot b + a\cdot c
\:\: \wedge \:\:
(b+c)\cdot a = b\cdot a + c\cdot a;
$$
$$
\exists 1\in A, \forall x\in A:\quad
x\cdot 1 = 1\cdot x = 1.
$$
Prove that $A$ is a ring with unit.
\end{ejercicio}

\begin{ejercicio}
\label{ej:ejercicior2}
Let $(A,+,\cdot)$ be a nontrivial ring with unit and $a,b,c\in A$.
\begin{my_enumerate}
\item Prove that if $a$ is invertible, then 
$$
ab = ac \: \Rightarrow \: b=c,
$$
and that $a$ has only one multiplicative inverse.
%
\item Prove that if $a^2 = 0$, then $a+1$ and $a-1$ are invertible. 
\item Prove that if $a$ and $b$ are invertible, then $ab$ is invertible.
\item 
\label{pto:ayuda1}
 Prove that $(A^\times,\cdot)$ is a group.
\end{my_enumerate}
\end{ejercicio}

\begin{ejercicio}
Let $(F,+,\cdot)$ be a field with $|F|=m\in \mathbb{N}$. Prove that
\begin{equation}
\forall x\in F\setminus \{0\}:\quad
x^{m-1} = 1.
\end{equation}
\end{ejercicio}

\begin{ejercicio}
Let $A$ be a commutative ring and $a,b\in A$. Prove that if $ab$ is invertible, then $a$ and $b$ are both invertible.
\end{ejercicio}


\begin{ejercicio}
Let $(A,+,\cdot)$ be a nontrivial ring and $a,b,c\in A$.
\begin{my_enumerate}
\item Prove that if $a\notin \{-1,1\}$ and $a^2=1$, then $a+1$ and $a-1$ are zero divisors
\item Prove that if $ab$ is a divisor of zero, then either $a$ or $b$ is a zero divisor.
\end{my_enumerate}
\end{ejercicio}

\begin{ejercicio}
Prove that in a nontrivial commutative ring with unit, a zero divisor cannot be invertible. 
\end{ejercicio}



\begin{ejercicio}
Consider $A = (\entire, \oplus, \odot)$ where
$$
a\oplus b = a+b-1,
\quad
a\odot b = ab - (a+b) +2.
$$
\begin{my_enumerate}
\item Prove that $A$ is a commutative ring with unit. Indicate the zero element, the unit, and the negative of an arbitrary $a$.
\item Is $A$ an integral domain?
\end{my_enumerate}
\end{ejercicio}

\begin{ejercicio}
Consider $A = (\rational \times \rational, \oplus, \odot)$ where
$$
(a,b)\oplus (c,d) = (a+c,b+d),
\quad
(a,b)\odot (c,d) = (ac-bd,ad+bc).
$$
\begin{my_enumerate}
\item Prove that $A$ is a commutative ring with unit. Indicate the zero element, the unit element, and the negative of an arbitrary $a$.
\item Prove that $A$ is a field and indicate the multiplicative inverse of an arbitrary nonzero element.
\end{my_enumerate}
\end{ejercicio}

\begin{ejercicio}
Consider Example \ref{cuadratico} and $A = Q(\sqrt{2})$.
\begin{my_enumerate}
\item Prove that $A$ is a commutative ring with unit. Indicate the zero element, the unit, and the negative of an arbitrary $a=x+y\sqrt{2}$.
\item Prove that $A$ is a field.
\end{my_enumerate}
\end{ejercicio}

\begin{ejercicio}
Verify that $\funciones$ satisfies all the axioms of a commutative ring with unit. Indicate the zero element and invertible elements. Describe the zero divisors in $\funciones$. 
Explain why $\funciones$ is neither a field nor an integral domain.
\end{ejercicio}

%\begin{ejercicio}
%What kind of algebraic structure is $(\matriz[n](\real),+,\cdot)$? Explain why %$\matriz[n](\real)$ is not an integral domain.
%\end{ejercicio}

\begin{ejercicio}
Let $\Omega \neq \emptyset$ be a set and consider $A = (\parts, \Delta, \cap)$.
\begin{my_enumerate}
\item Prove that $A$  is a commutative ring with unit.
\item Describe the zero divisors and the invertible elements of $A$.
\item Explain why $A$ is not an integral domain.
\item Give the tables for additiona and multiplication of $A$ for $\Omega = \{a,b,c\}$.
\end{my_enumerate}
\end{ejercicio}

\begin{ejercicio}
The set of \emph{quaternions}, $\quaternions$, can be seen as the elements of $\matriz[2](\complex)$ with the form
\begin{equation}
\label{eq:trium1}
\alpha = 
\begin{pmatrix}
a+bi & c+di \cr
-c+di & a-bi
\end{pmatrix},\quad a,b,c,d \in \real.
\end{equation}
\begin{my_enumerate}
\item Prove that $\quaternions$ endowed with the usual addition and multiplication of matrices is a non-commutative ring with unit.
\item Prove that $\alpha$ as given in \eqref{eq:trium1}
can be written, in \emph{standard notation}, as
\begin{equation}
\label{eq:fuimr1}
\alpha = a \mathbf{1} + b \mathbf{i} + c\mathbf{j}
+ d \mathbf{k},
\end{equation}
where
$$
\mathbf{1} = \begin{pmatrix} 1 & 0 \cr 0 & 1    \end{pmatrix},\quad
\mathbf{i} =
\begin{pmatrix} i & 0 \cr 0 & -i    \end{pmatrix},\quad
\mathbf{j}=
\begin{pmatrix} 0 & 1 \cr -1 & 0    \end{pmatrix},\quad
\mathbf{k} =
\begin{pmatrix} 0 & i \cr i & 0    \end{pmatrix}.
$$
\item Prove that
\begin{equation}
\mathbf{i}^2 = \mathbf{j}^2 =
\mathbf{k}^2 = \mathbf{1};
\end{equation}
\begin{equation}
\mathbf{i} \mathbf{j} = - \mathbf{j} \mathbf{i} = \mathbf{k},\quad
\mathbf{j} \mathbf{k} = - \mathbf{k} \mathbf{j} = \mathbf{i},\quad
\mathbf{k} \mathbf{i} = - \mathbf{i} \mathbf{k} = \mathbf{j}.
\end{equation}
\item The \emph{conjugate} and \emph{norm} of the quaternion $\alpha = a \mathbf{1} + b \mathbf{i} + c\mathbf{j}
+ d \mathbf{k}$ are, respectively,
\begin{eqnarray}
\overline{\alpha} = a \mathbf{1} - b \mathbf{i} - c\mathbf{j}
- d \mathbf{k},
\\
\|\alpha\| = \sqrt{a^2 + b^2 + c^2 + d^2}.
\end{eqnarray}
Prove that 
$$
\alpha \overline{\alpha} = \overline{\alpha} \alpha = \|\alpha\|^2 \mathbf{1}.
$$
\item Prove that $\quaternions$ is a \emph{skew field} i.e., it's a (not necessarily commutative) ring with unit in which every nonzero element has a multiplication inverse.
\end{my_enumerate}
\end{ejercicio}

\begin{ejercicio}
Let $G$ be an additive Abelian group. An endomorphism on $G$ is a homomorphism from $G$ into $G$. Prove that $\mathrm{End}(G)$, the set of endomorphisms on $G$ becomes a ring with unit when it's endowed with addition and the composition product.
\end{ejercicio}

\begin{ejercicio}
Let $(A,+,\cdot)$ be a ring. An element $a\in A$ is said to be \emph{nilpotent} if
$$
\exists n\in \mathbb{N}:\quad a^n = 0.
$$
\begin{my_enumerate}
\item Prove that if $A$ has a unit element and $a\in A$ is nilpotent, then both $a+1$ and $a-1$ are invertible.
\item Prove that if $A$ is commutative and $a\in A$ is nilpotent, then $xa$ is nilpotent, for all $x\in A$.
\item Prove that if $A$ is commutative and $a,b\in A$  are nilpotent, then $a+b$ is nilpotent.
\end{my_enumerate}
\end{ejercicio}

\end{questions}

\end{document}